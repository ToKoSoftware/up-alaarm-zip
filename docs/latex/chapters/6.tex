\section{Fazit}

Die vorliegende Projektdokumentation beschrieb die Schaffung einer Simulation von Alarmszenarien im Kontext des Themenfelds Industrie 4.0. 

Die Projektarbeit konzentrierte sich darauf, Alarmszenarien im Industrie-4.0-Kontext zu entwickeln und durch spezifische Komponenten die Immersion dieser Szenarien zu intensivieren. Die Integration eines Minimum Viable Products (MVP) in die Simulationsumgebung des Zentrums Industrie 4.0 ermöglicht es, Störfallszenarien realitätsnah zu simulieren und die Reaktionen der Teilnehmer zu beobachten. Dies eröffnet neue Möglichkeiten für die Mensch-Maschine-Interaktion und die Bewertung von Verhaltensweisen in Gefahrensituationen im Kontext zunehmend vernetzter Produktionsumgebungen.

Die entworfenen Alarmszenarien eröffnen nicht nur neue Forschungspotenziale, sondern können auch zur Entwicklung von Schulungsprogrammen und Übungen genutzt werden, um das Bewusstsein und die Fähigkeiten der Mitarbeiter im Umgang mit potenziellen Gefahren zu stärken. Zudem besteht anhand dieser Projektdokumentation und dem gegebenen \textit{git-Repository} die Möglichkeit, die Alarmsimulation durch neu zu definierende Szenarien und einzuführende Komponenten mit geringem Aufwand zu erweitern.
