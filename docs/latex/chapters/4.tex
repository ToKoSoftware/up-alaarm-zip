\newpage
\section{Konzeptionierung des finalen Szenarios}

Die Konzeptionierung unseres finalen Szenarios folgte einem stringenten Vorgehensansatz, der aus mehreren aufeinanderfolgenden Schritten bestand. Im initialen Schritt führten wir ein individuelles Brainstorming über die verschiedenen Komponenten und Abläufe der Szenarien durch. Hierbei bedienten wir uns digitaler \glqq Sticky Notes\grqq{} auf einem Miro Board. In diesem kreativen, ungebundenen und explorativen Prozess wurden diverse Aspekte in Betracht gezogen. Exemplarisch verdeutlichen einige Beispiele der \glqq Sticky Notes\grqq{} die Breite der Diskussionen und Ideenfindung. Unter anderem wurde die Idee einer Alarmsignalgabe in Form von Bass oder Sirene erörtert, die nur bei der jeweiligen Maschine ausgelöst werden konnte. Dieses Signal sollte in einer Art gestaltet sein, dass es sich pulsierend wiederholt, solange die Maschine nicht repariert wurde. Abhängig von der benötigten Stressstufe sollte sich sowohl die Tonhöhe als auch die Lautstärke des Geräusches verändern. Ein weiterer Vorschlag bestand darin, Brandgeruch als Warnsignal einzusetzen. Auch visuelle Effekte wie ein schnelles, abwechselndes Licht wurden als Möglichkeit in Betracht gezogen. Zusätzlich wurde die Simulation eines Stromausfalls durch zeitweises Abschalten von Licht und Displays diskutiert. Dies würde von einem FI-Schutzschalter ausgelöst.

Im darauffolgenden Schritt erfolgte die Kategorisierung der \glqq Sticky Notes\grqq{} in verschiedene Szenarien, Komponenten und nachfolgende Aktivitäten. Die Szenarien wurden in die Kategorien \glqq Riechen\grqq{}, \glqq Hören\grqq{}, \glqq Sehen\grqq{} und \glqq Fühlen\grqq{} unterteilt. Die Komponenten wurden wiederum in solche eingeteilt, die für den Einsatz im Szenario und solche, die zur Bewältigung der Situation notwendig waren. Um die Übersichtlichkeit und die Kommunikation zu verbessern, wurde dieser Schritt persönlich durchgeführt.

Anschließend überführten wir die physisch erstellte Übersicht in digitale Form. In einer Plenumsdiskussion bewerteten alle Mitglieder die präferierten \glqq Sticky Notes\grqq{}, indem sie Punkte vergaben. Auf diese Weise konnten wir eine gemeinsame Entscheidungsgrundlage schaffen und die wichtigsten Ideen identifizieren.

Zu guter Letzt selektierten wir drei Szenarien basierend auf den \glqq Sticky Notes\grqq{} mit den höchsten Punktzahlen. Das erste Szenario umfasste den Einsatz von visuellen Effekten mit flackerndem Licht, einem an das Stresslevel angepassten Geräusch, LED-Streifen und Lautsprechern. Im zweiten Szenario lag der Schwerpunkt auf dem Einsatz von Brandgeruch, begleitet von visuellen Effekten wie Scheinwerfern in rotem Licht, einer Infrarotlampe und dem Erzeugen von Rauch mithilfe einer Nebelmaschine. Auch hier wurden LED-Streifen verwendet. Das dritte Szenario beinhaltete Schläge oder Stöße, die den Bruch einer Maschine simulieren sollten. Das begleitende Geräusch sollte ebenfalls je nach Stresslevel variieren und wurde durch einen Subwoofer und Lautsprecher unterstützt.

Dieser systematische Ansatz bei der Erstellung des finalen Szenarios ermöglichte es uns, die verschiedenen Ideen zu erfassen, zu strukturieren und zu bewerten.

Aus dem Entwurf der drei Alarmszenarien haben sich abschließend diverse Vorgehensweisen und spezifische Komponenten manifestiert, die für die Realisierung des Projekts unabdingbar sind. Inbegriffen sind hierbei die Interaktionsplattformen sowie die damit einhergehende Technik, Entstörungsverfahren, Kommunikationsschemata und audiovisuelle Signaltechniken.

Die nun spezifizierten Szenarioansätze und die damit verknüpften Bestandteile wurden als BPMN-Modelle entworfen und Herrn Panzer zur Begutachtung vorgestellt. Die individuellen Szenariomodelle repräsentieren schematisch den Ablauf der Szenarien und ermöglichen eine mühelose Verfolgung der divergenten möglichen Verläufe, Resultate sowie ihrer Kombinationen und entsprechenden Auslöser.

Im Anschluss wurden die modellierten Szenarien in ein zentralisiertes Modell transferiert, um die besten Merkmale und Ansätze aller Szenarien zu kombinieren. Hierzu hat jedes Mitglied der Gruppe in den drei zugrunde liegenden Szenariomodellen für individuelle Komponenten Punktzahlen vergeben und gekennzeichnet. Diese Einschätzungen wurden einer eingehenden Auswertung unterzogen, wobei die Bestandteile mit den höchsten Punktzahlen in das Hauptmodell integriert wurden.

Die Gruppe hat sich für das Projekt auf wöchentliche Meetings verständigt, in denen Ergebnisse abgeglichen und korrigiert werden sollten, sowie die Zuweisung von einzelnen Tasks erfolgte. Die Meetings fanden jeweils mittwochs gegen 17:00 Uhr deutscher Zeit statt.
\\
\label{section:name-}