\section*{Anhang}
\label{sec:anhang}

\subsection*{Sicherheitsbriefing}\hspace{0pt}\marginpar{\footnotesize{ca. 2 Min.}}
\emph{Bitte nehmen Sie sich die Zeit, diese Anweisungen sorgfältig zu lesen und zu verstehen, bevor Sie mit dem Betrieb der Maschine fortfahren. Ihre Sicherheit hat oberste Priorität.}

\emph{WICHTIG: GESUNDHEITSWARNUNG} \\

\emph{Bitte stellen Sie sicher, dass Sie die geeignete Schutzkleidung tragen, bevor Sie mit dem Betrieb dieser Maschine beginnen. Dies beinhaltet Schutzhandschuhe, Schutzbrille und nach Anweisung des Übungsleiters eine Atemschutzmaske. 
Es besteht die Möglichkeit, dass während des Szenarios Rauch oder andere intensiv riechende Substanzen verwendet werden.
Dieses Szenario beinhaltet flackernde Lichter, die bei einigen Personen mit Epilepsie zu Anfällen führen können. Bitte beachten Sie, dass Ihre Gesundheit und Sicherheit oberste Priorität haben. Falls Sie Bedenken haben oder gesundheitliche Einschränkungen haben, die sich auf Ihre Teilnahme an diesem Szenario auswirken könnten, empfehlen wir Ihnen, vorab Rücksprache mit einem Arzt zu halten, um sicherzustellen, dass Sie angemessen geschützt sind und das Szenario ohne Risiken für Ihre Gesundheit durchführen können.}

\subsection*{Beschreibungskarte MVP}

In diesem Szenario wird ein Ablauf aus verschiedenen Komponenten und Eskalationsstufen beschrieben, das in der Simulationsumgebung des Zentrums Industrie 4.0 implementiert wurde. Die Szenariokomponenten bestehen aus den Teilnehmern, Tablet, Subwoofer/Lautsprecher, Wärmelampe/Licht und Nebelmaschine. Diese Komponenten sind in einer Umgebung platziert, in der es Anlagen zum Bedienen gibt, die alle durch ein Förderband miteinander verbunden sind. Das System ist so konzipiert, dass es drei Eskalationsstufen gibt, die nacheinander aktiviert werden können. Jede Eskalationsstufe beinhaltet einen Entstörungsvorgang, der vom Teilnehmer gelöst werden muss. Dabei werden verschiedene Szenariokomponenten aktiviert, um den Prozess zu unterstützen und zu simulieren. Teilnehmer des Szenarios, die bei einem Enstörungsvorgang versagen, werden automatisch zur nächsten Eskalationsstufe befördert, bis sie maximal die Eskalationsstufe 3 erreichen. Während eines Enstörungsvorgangs steht den Teilnehmern dabei nur ein Versuch zur Verfügung. Allerdings haben erfolgreiche Szenarioteilnehmer die Möglichkeit, nach Abschluss einer beliebigen Eskalationsstufe das gesamte Szenario vorzeitig zu beenden. Pro Durchgang führt nur ein Szenarioteilnehmer das Szenario durch. Es gibt einen Übungsleiter, der diese Vorgänge beobachtet und bei Notfällen einschreitet, jedoch ist er nicht für die gesamten Handlungen im Szenarioprozess eingeplant.

\

\underline{Einleitung zum Szenario:} \\
Der Teilnehmer aktiviert die Anlage mit einem Startknopf auf dem Tablet.\hspace{0pt}\marginpar{\footnotesize{ca. 10 Sek.}}

\

\

\underline{Einleitung der Eskalationsstufe 1:}

Die Anlage läuft an. \hspace{0pt}\marginpar{\footnotesize{ca. 8 Sek.}}
Gleichzeitig werden in der ersten Stufe die Szenariokomponenten aktiviert, wie der Subwoofer/Lautsprecher, der einen langsamen pulsierenden Ton erzeugt,\hspace{0pt}\marginpar{\footnotesize{ca. 1 Sek.}} sowie die Wärmelampe/Licht (Andon-Signalleuchte), die an der jeweiligen Maschine aktiviert wird und kontinuierlich leuchtet.
Die Anlage ruckelt und stoppt ihren Arbeitsfluss. \hspace{0pt}\marginpar{\footnotesize{ca. 5 Sek.}}

\

\subsection*{Anzeige der Fehlermeldung und Lösungsansatz}

\hspace{0pt}\marginpar{\footnotesize{ca. 40 Sek.}} Auf dem Tablet erscheint der Name der betroffenen Maschine, zu der der Szenarioteilnehmer hinlaufen muss:

\

\emph{FEHLERCODE: E5432}\\
\\

Hinlaufen zu dieser Maschine. \hspace{0pt}\marginpar{\footnotesize{ca. 15 Sek.}}



\




\hspace{0pt}\marginpar{\footnotesize{ca. 45 Sek.}}Auf dem Tablet erscheint ein QR-Code Scanner, der zum Scannen des QR Codes auf der Maschine benutzt werden muss, um einen Entstörungsvorgang in 40 Sekunden bzw. das Problem zu lösen. Eine Nachricht wird angezeigt, die besagt: "Dies ist der richtige Code! Starte Spiel...".
Wenn der gescannte Code nicht dem erwarteten Code entspricht: Eine rote Fehlermeldung wird angezeigt, die besagt: "Falscher Code! Gehen Sie zu Machine03." Nach einer kurzen Verzögerung von 2 Sekunden wird der QR-Code-Leser erneut aktiviert, um einen weiteren Scan zu ermöglichen.
\


\subsection*{Eskalationsstufe 1}

\

\emph{Auf dem Tablet wird eine Notiz angezeigt:} \\
\emph{(!) Die Werkschritt-Reihenfolge der Maschine scheinen durcheinander gebracht zu sein. } \\

\emph{Klicke auf die Nummern 1 bis 10 in aufsteigender Reihenfolge, um die richtige Reihenfolge wiederherzustellen. Für die Operation sind nur 10 Sekunden vorgesehen!} \\

\emph{3, 5, 8, 10, 9, 2, 7, 4, 1, 6} \\

\
Es folgt eine Verzweigung, die eine Zufallsvariable beinhaltet:

{Nach erfolgreichem Abschluss des Entstörungsvorgangs wird auf dem Tablet eine Benachrichtigung angezeigt, die den Abschluss des Vorgangs bestätigt. Um die Wahrscheinlichkeit von 33\% zu simulieren, wird eine digitale Würfelfunktion verwendet, die die Zahlen 1, 2 und 3 generiert und damit drei mögliche Zustände repräsentiert. Nach dem Entstörungsvorgang wird die digitale Würfelfunktion ausgeführt. Wenn die gewürfelte Zahl eine 1 oder 2 ist (entspricht einer Wahrscheinlichkeit von 2/3), wird auf dem Tablet angezeigt, dass das Problem als gelöst erfasst wurde. Falls die gewürfelte Zahl eine 3 ist (entspricht einer Wahrscheinlichkeit von 1/3), wird auf dem Tablet angezeigt, dass das Problem nicht vollständig gelöst wurde. \\

\begin{itemize}
\item
Sollte der Entstörungsvorgang innerhalb der vorgegebenen Zeit nicht erfolgreich abgeschlossen werden oder eine falsche Eingabe getätigt werden, erfolgt der Übergang zur zweiten Eskalationsstufe automatisch.
\item
Ist der Entstörungsvorgang erfolgreich gelöst, folgt die Zufallsvariable, dabei wird bei einer 1/3 Wahrscheinlichkeit das Problem als nicht gelöst erfasst. Dann erfolgt ebenfalls der Übergang zur zweiten Eskalationsstufe automatisch.
\item
\hspace{0pt}\marginpar{\footnotesize{ca. 3 Sek.}}Wenn der Entstörungsvorgang erfolgreich gelöst ist, folgt die Zufallsvariable, bei der mit einer Wahrscheinlichkeit von 2/3 das Problem als gelöst erfasst wird. Wenn dieses Szenario eintrifft, werden alle Szenariokomponenten ausgeschaltet und das Szenario für den Teilnehmer ist erfolgreich beendet.
\end{itemize}

\newpage 

\subsection*{Eskalationsstufe 2}

Simultan werden die Szenariokompontenten Subwoofer/Lautprecher mit einem mittelschnell schlagenden Ton aktiviert. Der Vaporizer stößt derweil Brandgeruch aus. Die LED-Leuchte flackert bei der Maschine blau. Diese bleiben durchgehend in der zweiten Eskalationsstufe aktiv.\hspace{0pt}\marginpar{\footnotesize{ca. 5 Sek.}} \\
Der Szenarioteilnehmer soll sich selbstständig zur nächsten Maschine bewegen, was durch eine Andon-Signalleuchte an der Maschine signalisiert wird. (Zeit: ca. 15 Sekunden) \\
\hspace{0pt}\marginpar{\footnotesize{ca. 40 Sek.}}Erneut muss ein Entstörungsvorgang durchgeführt werden: 

\emph{(!) Auf dem Tablet wird eine Notiz angezeigt:}\\

Der Prozessfluss der Maschine scheint ein Problem zu haben und zu überhitzen. Mehrere Prozesse scheinen separat zu laufen, die wieder in die richtige Reihenfolge gebracht werden müssen.
        Klicke die Nummern 1 bis 10 in aufsteigender Reihenfolge an, um dies zu tun.
        Nach jeweils 5 und 10 Sekunden werden die Prozessschritte durch eine Neukalibrierung gemischt.
        Für diese Reparatur sind aufgrund der engen Taktung lediglich 15 Sekunden vorgesehen. \\


\emph{3\hspace{15pt} 5\hspace{42pt} 8\hspace{58pt} 10\hspace{75pt} \\9\hspace{25pt} 2\hspace{37pt} 7\hspace{83pt} \\4\hspace{39pt} 1\hspace{66pt} 6\hspace{81pt}} \\

Danach erfolgt eine weitere Verzweigung, bei der die Zufallsvariable aktiviert wird. \\

\subsection*{Eskalationsstufe 3} \

Gleichzeitig werden die Szenariokomponenten Subwoofer/Lautsprecher mit einem schnell schlagenden Ton, der Vaporizer mit einem intensiveren Geruch als zuvor und die Wärmelampe aktiviert. Die Andon-leuchte flackert rot. Diese bleiben durchgehend in der dritten Eskalationsstufe aktiv.\hspace{0pt}\marginpar{\footnotesize{ca. 5 Sek.}} \\
Der Szenarioteilnehmer soll zur nächsten Maschine, welches durch eine Andon-Signalleuchte signalisiert wird, hinbewegen.\hspace{0pt}\marginpar{\footnotesize{ca. 15 Sek.}} \\
Erneut muss ein Entstörungsvorgang durchgeführt werden: \hspace{0pt}\marginpar{\footnotesize{ca. 40 Sek.}}

\emph{Auf dem Tablet wird eine Notiz angezeigt:} \\

\emph{(!) Die Werkschritt-Reihenfolge der Maschine scheinen durcheinander gebracht zu sein. } \\

\emph{Etwas scheint mit der Kalibrierung der Maschine ein Problem zu geben, wodurch schwerwiegende Fehler auslöst wurden.
                        Durch eine Fehlkalibrierung ist es zu einem Kabelbrand gekommen. Um den Schaden zu begrenzen, müssen die Eingaben schnellstmöglich neu kalibriert werden, um einen Brand zu verhindern.
                        Zur korrekten Kalibrierung muss der richtige grün aufleuchtende Bereich innerhalb von 0,6 Sekunden gedrückt werden. 
                        Die Bereiche fangen in einem schwarzen Zustand an. Falls du länger als 0,6 Sekunden brauchst, 
                        um den grün aufleuchtenden Bereich zu drücken, kann die Maschine nicht richtig kalibriert werden.
                        Es werden 6 erfolgreiche Kalibrierungen benötigt.} \\

\
Danach erfolgt eine weitere Verzweigung, bei der die Zufallsvariable aktiviert wird.

\subsection*{Voraussichtliche Zeitangaben des Szenarios}
Sämtliche Zeitangangaben beruhen auf einer Schätzung bzw. theoretischen Planung und sind je nach individueller Durchführung und Reaktionsgeschwindigkeit des Teilnehmers variabel. Diese Zeitangaben stehen daher unter Vorbehalt und dienen lediglich als grobe Orientierung. Es ist möglich, dass die tatsächliche Dauer für jeden Teilnehmer unterschiedlich ist. \\

\

\underline{Benötigte Zeit für Eskalationsstufe 1} \\
Lesen der Anweisungen: ca. 2 Minuten \\
Aktivierung der Anlage: ca. 8 Sekunden \\
Unterbrechung der Anlage ca. 5 Sekunden \\
Anzeige der Fehlermeldung: ca. 40 Sekunden \\
Hinlaufen zur Maschine: ca. 15 Sekunden \\
Entstörungsvorgang in der ersten Stufe: ca. 45 Sekunden \\
Einschreiten der Zufallsvariable: ca. 3 Sekunden \\
Geschätzte Gesamtzeit für den Abschluss der ersten Stufe: ca. 4 Minuten \\

\

\underline{Benötigte Zeit für Eskalationsstufe 2} \\
Zeit für die erste Stufe: ca. 4 Minuten \\
Hinlaufen zur Maschine: ca. 15 Sekunden \\
Entstörungsvorgang in der zweiten Stufe: ca. 40 Sekunden \\
Einschreiten der Zufallsvariable: ca. 3 Sekunden \\
Geschätzte Gesamtzeit für den Abschluss der ersten und der zweiten Stufe: ca. 5 Minuten \\

\

\underline{Benötigte Zeit für Eskalationsstufe 3 / Max. Zeit bei Erfolglosigkeit} \\
Zeit für die zweite Stufe: ca. 5 Minuten \\
Hinlaufen zur Maschine: ca. 15 Sekunden \\
Entstörungsvorgang in der dritten Stufe: ca. 40 Sekunden \\
Einschreiten der Zufallsvariable: ca. 3 Sekunden \\
Geschätzte Gesamtzeit für den Abschluss der ersten, zweiten und der dritten Stufe: ca. 6 Minuten \\






















