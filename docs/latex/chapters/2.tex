\section{Hintergrund}

Im folgenden Kapitel werden verwandte Themenbereiche zu Alarmszenarien und Alarmsimulationen vorgestellt. Zunächst werden die Rechercheergebnisse zu kommerziellen Lösungen für Alarmszenarien (Kapitel \ref{chap:commercial_research}) behandelt. Anschließend wird ein Überblick über Übungsszenarien (Kapitel \ref{chap:standart_research}) gegeben, gefolgt von einer Übersicht zur wissenschaftlichen Fachliteratur  (Kapitel \ref{chap:scientific_research}) zum gleichen Themenbereich.

\subsection{Kommerzielle Lösungen}
\label{chap:commercial_research}
Im Branchenumfeld des Maschinenbaus existiert eine Vielzahl von \textbf{kommerziellen Alarmsystemlösungen}. Im Folgenden sind einige dieser Angebote aufgeführt, die durch zusätzliche Hardware Echtzeitüberwachung ermöglichen:

\begin{description}
    \item [\textbf{HGP-Eberle:}] 
    Ein Cloud-basiertes Alarmierungssystem mit permanenter Überwachung durch eine direkt an die Maschine angeschlossene Alarm-Box. Bei Störungen benachrichtigt das System zuständiges Personal mittels einer App. (\cite{HGP-Eberle})
   
   \item [\textbf{Ixon-Cloud:}] 
   Webbasiertes Alarmierungssystem, welches Empfänger weltweit alarmiert. Maschinen werden kontinuierlich durch ein Edge-Gerät überwacht, wodurch Datenanalysen für Dashboards und Reports ermöglicht werden. (\cite{Ixon-Cloud})
   
   \item [\textbf{Mobeye:}] 
   Ein Gerät überwacht die Stromversorgung der Maschine. Im Falle einer Störung wird eine Alarmbenachrichtigung per App oder E-Mail versendet. Dabei können Nutzer zwischen verschiedenen Benachrichtigungsabläufen wählen. (\cite{Mobeye})
   
   \item [\textbf{Totmannschalter:}] 
    Ein Gerät zur Überwachung von allein arbeitenden Personen. Bei Handlungsunfähigkeit oder durch manuellen Auslöser wird ein Notruf versendet, wobei die Person via GPS geortet wird. (\cite{totmann})
\end{description}
 
Des Weiteren unterscheiden sich folgende kommerzielle Benachrichtigungssysteme:

\begin{description}
   \item [\textbf{Workerbase:}] Diese App zeigt Maschinenausfälle als Warnung auf Endgeräten ausgewählter Mitarbeiter an. Dabei sind unterschiedliche Alarmierungs- und Benachrichtigungsfunktionen verfügbar. Die im Workerbase-System gesammelten Daten können über die App analysiert werden.(\cite{Workerbase})
   
   \item [\textbf{safeREACH:}] Durch diese App können alle betroffenen Personen in verschiedenen Notfallsituationen kontaktiert werden. Das Auslösen eines Alarms erfolgt manuell oder automatisch durch Drittsysteme. Nach der Alarmierung stehen Tools für angemessene Maßnahmen bereit. Alle Aktivitäten werden automatisch protokolliert. (\cite{safeREACH})
   
   \item [\textbf{Vedosign:}] Per Knopfdruck ausgelöste Alarme senden Textnachrichten an nahegelegene mobile Geräte, um den passenden Mitarbeiter für den Alarm zu benachrichtigen. (\cite{vedosign})
   
   \item [\textbf{Everbridge:}] Diese Software versendet Nachrichten an spezifische Personengruppen oder Bevölkerungsteile in Gefahrenzonen. (\cite{everbridge})

   \item [\textbf{Gisbo:}] Diese Alarmierungssoftware unterstützt das Krisenmanagement in Gefahrensituationen. Verschiedene Alarmarten, sowohl akustisch als auch optisch, können über die bestehende IT-Infrastruktur weitergeleitet werden.(\cite{Gisbo})

   \item [\textbf{WEKA:}] Elektroakustische Notfallsysteme gemäß DIN EN 50849 warnen Personen vor Ort und fordern zur Selbstrettung auf. Unterschiedliche Signaltöne und gespeicherte Alarmtexte werden durch Sirenen übermittelt. (\cite{WEKA})
   
   \item [\textbf{Videc:}] Dieses System ermöglicht sofortige und zielgerichtete Benachrichtigungen per E-Mail, SMS, Sprachnachricht, Audio, Messenger oder Pager an beteiligte Empfänger. (\cite{VIDEC})

   \item[ISA Telematics:] Das Angebot dieses Anbieters umfasst die Sicherheits-App ''iTProtection'', worüber Notrufe ausgelöst und Personen geortet werden können, und die Telematikplattform ''iTelematics HL'', auf der Alarmmeldungen verwaltet werden. Nutzer erhalten akustische Meldungen und können auf Alarmpläne oder medizinische Informationen zugreifen. (\cite{ISA_Telematics})

\end{description} \ 

\newpage
\subsection{Simulation von Alarmszenarien}
\label{chap:standart_research}
In verschiedenen Zusammenhängen wurden Richtlinien, Handbücher und Empfehlungen zum \textbf{Inhalt und zur Durchführung von Sicherheitsübungen} festgehalten. \\
Grundsätzlich wird dabei zwischen zwei Arten von Übungen unterschieden:
\begin{itemize}
    \item \textbf{Simulations- oder Stabsrahmenübungen: } Üben von Führungsfunktionen und Aufgaben von Einsatzkräften in geschlossenen Räumen (→ strategische Entscheidungen). Diese Übungen sind mit weniger Aufwand verbunden und dadurch realitätsferner.   
    \item \textbf{Vollübungen: }Reales Handeln der Übungsteilnehmer. Das gesamte Einsatzgeschehen wird innerhalb der Übung möglichst realitätsnah abgebildet.
\end{itemize}\ 


Der „Leitfaden für strategische Krisenmanagement-Übungen“ vom \textbf{Bundesamt für Bevölkerungsschutz und Katastrophenhilfe} (\cite{strat_KrisenMGMT}) soll der effektiven Vorbereitung auf eine Krisensituation dienen. Angewendet werden die Inhalte unter anderem bei der länder- und ressortübergreifende Krisenmanagementübung (LÜKEX), die in regelmäßigen Abständen in Deutschland durchgeführt wird. \\
Der Ablauf der strategischen Krisenübung lässt sich in vier Abschnitte unterteilen:

\begin{itemize}
    \item \underline{Übungsplanung:}
    Es erfolgt eine konzeptionelle Vorbereitung. Im Übungsrahmen, dem zentralen Dokument, werden das Thema, die Beteiligten, Ziele, Organisationseinheiten und Kosten der Übung festgelegt. Es wird ein grobes Übungsszenario entwickelt und die Grundzüge der Übungsauswertung werden gesetzt.  \
    
    \item \underline{Übungsvorbereitung:}
    Es erfolgt der Aufbau der Übungssteuerung, des Kommunikationsplans und es wird das Konzept des Übungsszenarios weiterentwickelt. Dabei wird ein Drehbuch entwickelt, welches den chronologischen Verlauf der Übung festhält. Die Kernelemente des Szenarios sind dabei der allgemeine Hintergrund, die fiktive Ausgangslage, die fiktive Lageentwicklung sowie jegliche Annahmen und Besonderheiten. Des Weiteren wird ein Auswertungskonzept inklusive Auswertungsunterlagen entwickelt und die Inhalte der Vorbesprechung sowie Anweisungen für Teilnehmenden werden bestimmt.\

    \item \underline{Übungsdurchführung:}
    Vor der Durchführung ist eine Abstimmung aller Maßnahmen essenziell, um mögliche Missverständnisse zu vermeiden. Es erfolgt anschließend eine Lageeinweisung aller Beteiligten in ihre Rollen und die Überprüfung der Technik. Im Anschluss kann die geplante Übung durchgeführt und dokumentiert werden. 
    Bei frei verlaufenden Übungen sollte keine Korrektur der Entscheidungen erfolgen und die Übung sollte auch bei abweichendem Verlauf nicht unterbrochen werden, solange die fiktive Lage und das Szenario beachtet werden.
\

    \item \underline{Übungsauswertung:}
    Nach dem Durchlauf der Übung werden Erfahrungsberichte, Dokumentation, Fragebögen und Berichte von Beobachtern in einem zentralen Workshop behandelt und ein Auswertungsbericht erstellt.\\    

\end{itemize} 


Der Behelf für das „Anlegen und Durchführen von Einsatzübungen“ des \textbf{Bundesamts für Bevölkerungsschutz der Schweizerischen Eidgenossenschaft} \autocite{Behelf_Einsatzübungen} ist eine Unterlage, die zur Ausbildung dient und neben der Anleitung auch Formulare und Textvorlagen zur Verfügung stellt. \\
Der erste Teil der Anleitung behandelt das Anlegen einer Übung. Es wird der Bedarf für eine Übung ermittelt und es werden Thema, Ziele, Gelände sowie Übungsobjekte der Übung bestimmt. Anschließend wird ein Konzept erstellt. Das Konzept soll auf Ausführbarkeit überprüft und je nach Bedarf nur als Konzept oder auch mit detaillierteren Unterlagen dokumentiert werden. \\
Der zweite Teil geht auf die Durchführung einer Übung ein. Die Übungsleitung wird in klar zugewiesene Aufgabenbereiche eingeteilt, wobei die Beteiligten die nötige Fachkompetenz aufweisen sollten. Kommunikationsmittel und genutzte Markierungen/Signaturen müssen festgelegt sein. Auf das Durchführen einer Übung folgt eine Besprechung, die eine Bilanzierung, Erläuterung von Zusammenhängen, Bewertung und Würdigung der Arbeit sowie darauffolgende künftige Lehren behandeln soll. Die durchgeführte Übung wird anschließend mit der AEK-Methode (Aussage-Erkenntnis-Konsequenz) ausgewertet. \ 


Das \textbf{Deutsche Rote Kreuz} bietet im Buch „Durchführung und Auswertung von MANV-Übungen“ \autocite{MANV-Übungen} eine Bandbreite an Konzepten und Umsetzungshilfen für Übungen in Bezug auf das Szenario des Massenanfalls von Verletzten (MANV).
Auch hier beginnt der Übungsablauf mit der Planung. Ziele, Schadenslage, Einsatzmittel, Verlauf und das Budget werden festgelegt. \\
In der anschließenden Vorbereitungsphase werden die Beteiligten benannt und die Kommunikation abgestimmt. Da bei dieser Art von Übung eine hohe Zahl an „Mimen“ (Figuranten, die im Szenario verletzte Personen darstellen) nötig ist, erfolgt in diesem Schritt die Registrierung, Terminabklärung sowie die Erstellung eines Zeitplans für den Übungstag. \\

\ 

Im nächsten Schritt folgt die Durchführung der Übung. Es erfolgt eine Einweisung und Sicherheitsbelehrung für die Beteiligten. Empfohlen werden pro Übung zwei Übungsläufe. \\
Nach der Durchführung erfolgt erst eine direkte Nachbereitung, bestehend aus einer Selbsteinschätzung und einer Vorstellung der Bewertungsindikatoren und Übungsdaten. Zuletzt folgt die spätere Nachbereitung, bei der die Übungsdaten in einer Führungskräftenachbesprechung evaluiert werden und anschließend ein Abschlussbericht an alle Beteiligten versendet wird. \\

Das Begleitheft zur Unterstützung der Unfallverhütung beim Übungs- und Schulungsdienst der \textbf{Arbeitsgemeinschaft der Feuerwehr-Unfallkassen} \autocite{Feuerweh-Unfallkasse} beschreibt einen empfohlenen Ablauf einer Einsatzübung. Dieser Ablauf gleicht im Allgemeinen den bereits aufgeführten Inhalten. \\
Es erfolgt eine Übungsvorbereitung mit Gefährdungsbeurteilung, technischer und organisatorischer Planung, und Vorbesprechung. Darauf folgt die Durchführung und anschließend wird eine Nachbereitung mit Nachbesprechung und Auswertung durchgeführt. \\ 
Zusätzlich sind Hinweise bezgl. der Übungsdurchführung genannt: \\
Der Übungsort soll auf Gefahrenquellen überprüft und ausreichend beleuchtet sein, die Ausrüstung soll vor Beginn überprüft werden, Übungsteilnehmer sollen nicht überlastet werden und auf besonders gefährliche Handlungen soll verzichtet werden. Der Übung soll dabei dieselbe Aufmerksamkeit wie einem realen Einsatz geschenkt werden. \\
Häufige Fehler treten bei Einsatzübungen durch mangelnde Vorbereitung, falsche Annahmen, fehlender Berücksichtigung von vorhandenem Fachwissen und Einbeziehung nicht relevanter Faktoren auf. Außerdem werden oft Aufgabenbereiche im Vorhinein zu detailliert erläutert und Darstellungen unzureichend bzw. nicht erkennbar einbezogen. \autocite{Feuerweh-Übungspräsentation}

Ein weiterer Anwendungsbereich, in dem Sicherheitsübungen erfolgen, ist das Fahrsicherheitstraining. Der Ablauf in der Praxis einschließlich des behandelten Inhalts, ist abhängig vom Veranstalter und Fahrzeugtyp. Zum Großteil besteht diese Übung aus einem Theorie-Teil, einem Praxis-Teil und einer Nachbesprechung. \autocite{DVR}

\newpage
\subsection{Wissenschaftliche Fachliteratur}
\label{chap:scientific_research}
Gegenwärtig gibt es Forschungsprojekte,  die immersive Software für das Training bzw. die Vorbereitung auf kritische Situationen getestet haben. Es gibt beispielsweise bereits eine Lehrsoftware für Grundschullehrer, die mittels virtueller Realität das richtige Verhalten in Brandsituationen schulen soll. Die Software beinhaltet verschiedene Lernmechanismen wie dynamische Geschichten, Realismus, Bewegungsfreiheit, Level- und Punktsysteme \autocite{DDE_of_VR}. In der Fertigungsindustrie gibt es bereits Alarm-Management-Systeme, die in VR-Anwendungen integriert wurden. Das System verbessert die Handhabung von Alarmen und soll dadurch die Sicherheit von Arbeitnehmern erhöhen. Das System kann Alarme automatisch auslösen und zeigt Benutzern visuelle Warnungen an, wenn ein Alarm ausgelöst wird \autocite{Design_of_VR-training}.

Darüber hinaus bestehen wissenschaftliche Studien zu VR/AR-Anwendungen, welche untersuchen, ob und inwieweit sich die Vorbereitung auf Notfälle, die Reaktion während eines Notfalls und die Erholung nach einem Notfall durch den Einsatz von VR und AR-Anwendungen verbessert \autocite{VRandAR}. 

Einige Produktionsmaschinen und Systeme sind so vernetzt, dass sie Maschinendaten selbständig auswerten und in der Cloud als 3D-Modell visualisieren können. Die Maschine gibt also selbst Rückmeldungen über ihren Status und kann ggf. Abweichungen und Fehler autonom melden \autocite{Mascienenausf_entdecken}.

Ein weiterer Ansatz besteht darin, die Gerätewartung und Diagnose mittels Augmented Reality zu verbessern. Durch die Nutzung des Systems konnten Studierende Datenanalysen und -erfassung für Wartungsanwendungen leichter ausführen. Das Verständnis der Studierenden in diesem Sachbereich wurde durch die Einbindung von AR positiv beeinflusst (\cite{Develop_and_Asses_AR}). 

\newpage




\newpage

%\subsection{Literaturverweise}
%[1] Mystakidis, S., Besharat, J., Papantzikos, G., Christopoulos, A., Stylios, C., Agorgianitis, S. and Tselentis, D. (2022). Design, Development, and Evaluation of a Virtual Reality Serious Game for School Fire Preparedness Training. Education Sciences, 12(4), p.281. doi:https://doi.org/10.3390/educsci12040281. \\

%[2] Matsas, E. and Vosniakos, G. (2015). Design of a virtual reality training system for human–robot collaboration in manufacturing tasks. [online] Available at: https://www.researchgate.net/publication/271837242_Design_of_a_virtual_reality_training_s ystem_for_human-robot_collaboration_in_manufacturing_tasks [Accessed 18 May 2023]. \\


%[3] Zhu, Y. and Li, N. (2021). Virtual and Augmented Reality Technologies for Emergency Management in the Built Environments: A State-of-the-Art Review. Journal of Safety Science and Resilience, Volume 2(March 2021). doi:https://doi.org/10.1016/j.jnlssr.2020.11.004. \\


%[4] Weidle, D. (2023). Maschinenausfälle entdecken bevor sie auftreten. Maschinenausfälle entdecken bevor sie auftreten, 26(5), pp.26–27. doi:https://doi.org/10.1002/citp.202300515. \\

%[5] Shyr, W.-J., Tsai, C.-J., Lin, C.-M. and Liau, H.-M. (2022). Development and Assessment of Augmented Reality Technology for Using in an Equipment Maintenance and Diagnostic System. Sustainability, 14(19), p.12154. doi:https://doi.org/10.3390/su141912154. \\


%[6] Yan, W., Wang, J., Lu, S., Zhou, M. and Peng, X. (2023). A Review of Real-Time Fault Diagnosis Methods for Industrial Smart Manufacturing. Processes, 11(2), pp.369–369. doi:https://doi.org/10.3390/pr11020369.
