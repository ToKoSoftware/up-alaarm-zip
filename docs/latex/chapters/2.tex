\section{Recherche}

\subsection{Recherche nach kommerziellen Lösungen}
Im Bereich des Maschinenbaus sind vielerlei \textbf{kommerzielle Lösungen} von Alarmsystemen zu nennen. Die folgenden Produktangebote und -hersteller bieten durch zusätzliche Hardware eine Echtzeitüberwachung an:
\begin{description}
    \item [\textbf{HGP-Eberle:}] 
    Cloud Alarmierungssystem mit permanenter Überwachung durch eine an die Maschine angeschlossene Alarm-Box. Im Störungsfall wird zuständiges Personal sofort per App benachrichtigt. (\cite{HGP-Eberle})
   
   \item [\textbf{Ixon-Cloud:}] 
   Webbasiertes Alarmierungssystem alarmiert Empfänger überall auf der Welt. Maschinen werden dabei durch ein Edge-Gerät kontinuierlich vor Ort überwacht und es wird eine Datenanalyse durchgeführt. (\cite{Ixon-Cloud})
   
   \item [\textbf{Mobeye:}] 
   Durch ein Gerät wird die Stromversorgung der Maschine überwacht. Bei einer Störung wird eine Alarmbenachrichtigung per App oder durch andere mobile Kommunikationswege versendet. Bei Push-Nachrichten lässt sich zudem zwischen verschiedenen Benachrichtigungsabläufen wählen. (\cite{Mobeye})
   
   \item [\textbf{Totmannschalter:}] 
   Ein Gerät überwacht eine allein arbeitende Person. Die Person kann einen manuellen Alarm auslösen, um einen Notruf abzugeben. Stellt das Gerät fest, dass die Person nicht handlungsfähig ist, wird automatisch ein Notruf ausgelöst. Per GPS-Ortung wird die Person lokalisiert. (\cite{totmann})
\end{description}
 
\

Des Weiteren ist zwischen den folgenden kommerziellen Benachrichtigungssystemen zu unterscheiden:
\begin{description}
   \item [\textbf{Workerbase:}] Die Maschinenalarm-App zeigt Maschinenausfälle als Warnung auf persönlichen Geräten der Mitarbeiter an. Dabei existieren verschiedene Alarmierungs- und Benachrichtigungsfunktionen für unterschiedliche Störungs- und Notfallsituationen. Mit den im Workerbase-System gesammelten Daten können über die App weitere Datenanalysen durchgeführt werden. (\cite{Workerbase})
   
   \item [\textbf{safeREACH:}] Über eine App können in verschiedenen Notfallsituationen alle betroffenen Personen kontaktiert und mit Informationen versorgt werden. Das Auslösen eines Alarms erfolgt per Knopfdruck oder durch Einbindung von Drittsystemen automatisch. Nach der Alarmierung stehen zudem weitere Tools zur Verfügung, um die Situation mit den richtigen Maßnahmen zu bewältigen. Die Aktivitäten werden automatisch protokolliert. (\cite{safeREACH})
   
   \item [\textbf{Vedosign:}] Durch einen per Knopfdruck ausgelösten Alarm wird eine standortbezogene Textnachricht an diverse mobile Geräte versendet. Die zielgerichtete Nachricht erlaubt die Benachrichtigung des richtigen Mitarbeiters, um präziser auf Alarme zu reagieren. (\cite{vedosign})
   
   \item [\textbf{Everbridge:}] Die Software macht es möglich, in sehr kritischen Gefahrensituationen eine Massennachricht über mehr als 25 Kontaktwege zu versenden. (\cite{everbridge})

   \item [\textbf{Gisbo:}] Diese Alarmierungssoftware unterstützt das Krisenmanagement in Gefahrensituationen. Verschiedene Alarmtypen können ausgelöst und Informationen schnell kommuniziert werden. Die Meldungen können akustisch und optisch sein, und werden über vorhandene IT-Infrastrukturen weitergeleitet. (\cite{Gisbo})

   \item [\textbf{WEKA:}] Elektroakustische Notfallsysteme (nach DIN EN 50849) warnen in Notfallsituationen anwesende Personen und fordern zu Selbstrettung auf. Durch unterschiedliche Signaltöne, die durch bspw. Sirenen übertragen werden, können Prioritäten im Vorfeld festgelegt und gespeicherte Alarmtexte wiedergegeben werden. (\cite{WEKA})
   
   \item [\textbf{Videc:}] Über das System werden sofortige und gezielte Benachrichtigungen an beteiligte Empfänger per E-Mail, SMS, Sprachnachricht, Audio, Messenger oder Pager versendet. Die Software fügt sich dabei in die vorhandene IT-Architektur des Unternehmens ein. (\cite{VIDEC})

   \item[ISA Telematics:] Das Angebot des Anbieters für Personen-Notsignal-Anlagen und Alleinarbeiterschutz umfasst die Sicherheits-App iTProtection, über welche Notsignale ausgelöst und Personen geortet werden können, und  die Telematikplattform iTelematics HL, über die Alarmmeldungen verwaltet werden können. Der Anwender wird über die Plattform akustisch über eine Alarmmeldung informiert und kann nach Anerkennung der Meldung auf hinterlegte Alarmpläne oder die Person betreffende medizinische Informationen zurückgreifen. (\cite{ISA_Telematics})

\end{description} \ 

\newpage
\subsection{Recherche nach Übungsszenarien und Standardabläufen}

In verschiedenen Zusammenhängen wurden Richtlinien, Handbücher und Empfehlungen zum \textbf{Inhalt und der Durchführung von Sicherheitsübungen} festgehalten. \\
Grundsätzlich wird dabei zwischen zwei Arten von Übungen unterschieden:
\begin{itemize}
    \item \textbf{Simulations- oder Stabsrahmenübungen: } Üben von Führungsfunktionen und Aufgaben von Einsatzkräften in geschlossenen Räumen (->strategische Entscheidungen). Diese Übungen sind mit weniger Aufwand verbunden, dadurch aber realitätsferner.   
    \item \textbf{Vollübungen: }Reales Handeln der Übungsteilnehmer. Das gesamte Einsatzgeschehen wird innerhalb der Übung möglichst realitätsnah abgebildet.
\end{itemize}\ 


Der „Leitfaden für strategische Krisenmanagement-Übungen“ vom \textbf{Bundesamt für Bevölkerungsschutz und Katastrophenhilfe} (\cite{strat_KrisenMGMT}) soll der effektiven Vorbereitung auf eine Krisensituation dienen. Angewendet werden die Inhalte unter anderem bei der Länder- und Ressortübergreifende Krisenmanagementübung (LÜKEX), die in regelmäßigen Abständen in Deutschland durchgeführt wird. \\
Der Ablauf einer solchen strategischen Krisenübung lässt sich in vier Abschnitte unterteilen:

\begin{itemize}
    \item \underline{Übungsplanung:}
    Es erfolgt eine konzeptionelle Vorbereitung. Im Übungsrahmen, dem zentralen Dokument, wird das Thema, die Beteiligten, Ziele, Organisationseinheiten und Kosten der Übung festgelegt. Es wird ein grobes Übungsszenario entwickelt und die Grundzüge der Übungsauswertung werden gesetzt.  \
    
    \item \underline{Übungsvorbereitung:}
    Es erfolgt der Aufbau der Übungssteuerung, des Kommunikationsplans und es wird das Konzept des Übungsszenarios weiterentwickelt. Dabei wird ein Drehbuch entwickelt, welches den chronologischen Verlauf der Übung festhält. Die Kernelemente des Szenarios sind dabei der allgemeine Hintergrund, die fiktive Ausgangslage, die fiktive Lageentwicklung, und jegliche Annahmen und Besonderheiten. Des Weiteren wird ein Auswertungskonzept inkl. Auswertungsunterlagen entwickelt und die Inhalte der Vorbesprechung und Anweisungen für Teilnehmende bestimmt.\

    \item \underline{Übungsdurchführung:}
    Vor der Durchführung ist eine Abstimmung aller Maßnahmen essenziell, um Missverständnisse zu vermeiden. Es erfolgt anschließend eine Lageeinweisung aller Beteiligten in ihre Rollen und die Überprüfung der Technik. Im Anschluss kann die geplante Übung durchgeführt und dokumentiert werden. 
    Bei frei verlaufenden Übungen soll keine Korrektur der Entscheidungen erfolgen und die Übung soll auch bei abweichendem Verlauf nicht unterbrochen werden, solange die fiktive Lage und das Szenario beachtet werden.
\

    \item \underline{Übungsauswertung:}
    Nach dem Durchlauf der Übung werden Erfahrungsberichte, Dokumentation, Fragebögen und Berichte von Beobachtern in einem zentralen Workshop behandelt und ein Auswertungsbericht erstellt.\\    

\end{itemize} 


Der Behelf für das „Anlegen und Durchführen von Einsatzübungen“ des \textbf{Bundesamts für Bevölkerungsschutz der Schweizerischen Eidgenossenschaft} \autocite{Behelf_Einsatzübungen} ist eine Unterlage, die zur Ausbildung dient und neben der Anleitung auch Formulare und Textvorlagen zur Verfügung stellt. \\
Der erste Teil der Anleitung behandelt das Anlegen einer Übung. Es wird der Bedarf für eine Übung wird ermittelt und es werden Thema, Ziele, Gelände sowie Übungsobjekte der Übung bestimmt. Anschließend wird das Konzept erstellt. Das Konzept soll auf Ausführbarkeit überprüft, und je nach Bedarf nur als Konzept oder auch mit detaillierteren Unterlagen dokumentiert werden. \\
Der zweite Teil geht auf die Durchführung einer Übung ein. Die Übungsleitung wird mit klar zugewiesenen Aufgabenbereichen eingeteilt, wobei die Beteiligten die nötige Fachkompetenz aufweisen sollten. Kommunikationsmittel und genutzte Markierungen/Signaturen müssen festgelegt sein. Dem Durchführen einer Übung folgt eine Besprechung, die eine Bilanzierung, Erläuterung von Zusammenhängen, Bewertung und Würdigung der Arbeit, und darauffolgende künftige Lehren behandeln soll. Die durchgeführte Übung wird anschließend mit der AEK-Methode (Aussage-Erkenntnis-Konsequenz) ausgewertet. \ 


Das \textbf{Deutsche Rote Kreuz} bietet im Buch „Durchführung und Auswertung von MANV-Übungen“ \autocite{MANV-Übungen} eine Bandbreite an Konzepten und Umsetzungshilfen für Übungen in Bezug auf das Szenario des Massenanfalls von Verletzten (MANV).
Auch hier beginnt der Übungsablauf mit der Planung. Ziele, Schadenslage, Einsatzmittel und Verlauf, und das Budget werden festgelegt. \\
In der anschließenden Vorbereitung werden Beteiligte benannt und die Kommunikation abgestimmt. Da bei dieser Art von Übung eine hohe Zahl an „Mimen“ (Figuranten, die im Szenario verletzte Personen darstellen) nötig ist, erfolgt in diesem Schritt die Registrierung und Terminabklärung, sowie die Erstellung eines Zeitplans des Übungstages. \\

\ 

Im nächsten Schritt folgt die Durchführung der Übung. Es erfolgt eine Einweisung und Sicherheitsbelehrung für die Beteiligten. Empfohlen werden pro Übung zwei Übungsläufe. \\
Nach der Durchführung erfolgt erst eine direkte Nachbereitung, bestehend aus einer Selbsteinschätzung und einer Vorstellung der Bewertungsindikatoren und Übungsdaten. Zuletzt folgt die spätere Nachbereitung, bei der die Übungsdaten in einer Führungskräftenachbesprechung evaluiert werden, und anschließend ein Abschlussbericht an alle Beteiligten versendet wird. \\

Das Begleitheft zur Unterstützung der Unfallverhütung beim Übungs- und Schulungsdienst der \textbf{Arbeitsgemeinschaft der Feuerwehr-Unfallkassen} \autocite{Feuerweh-Unfallkasse} nennt ebenfalls einen empfohlenen Ablauf einer Einsatzübung, der sich im Allgemeinen mit den bereits aufgeführten Inhalten gleicht. \\
Es erfolgt eine Übungsvorbereitung mit Gefährdungsbeurteilung, technischer und organisatorischer Planung, und Vorbesprechung. Darauf folgt die Durchführung und anschließend wird eine Nachbereitung mit Nachbesprechung und Auswertung durchgeführt. \\ 
Zusätzlich sind Hinweise bez. der Übungsdurchführung genannt: \\
Der Übungsort soll auf Gefahrenquellen überprüft und ausreichend beleuchtet sein, die Ausrüstung soll vor Beginn überprüft werden, Übungsteilnehmer sollen nicht überlastet werden und auf besonders gefährliche Handlungen soll verzichtet werden. Der Übung soll dabei dieselbe Aufmerksamkeit wie einem realen Einsatz geschenkt werden. \\
\textbf{Häufige Fehler} treten bei Einsatzübungen oft durch mangelnde Vorbereitung, falsche Annahmen, fehlender Berücksichtigung von vorhandenem Fachwissen und Einbeziehung nicht relevanter Faktoren auf. Außerdem werden oft Aufgabenbereiche im Vorhinein zu detailliert erläutert und Darstellungen unzureichend bzw. nicht erkennbar einbezogen. \autocite{Feuerweh-Übungspräsentation}

\ 

Ein weiterer Anwendungsbereich, indem Sicherheitsübungen erfolgen, ist das \textbf{Fahrsicherheitstraining}. Zwar ist Ablauf und Inhalt abhängig vom Veranstalter und Fahrzeugtyp, zum Großteil bestehen solche Übungen aus einem Theorie-Teil, einem Praxis-Teil und einer Nachbesprechung. \autocite{DVR}

\newpage
\subsection{Recherche nach wissenschaftlichen Studien}

\textbf{Es existieren bisher einige Projekte}, die immersive Software für das Training bzw. die Vorbereitung auf kritische Situationen getestet haben. So existiert bereits eine Lehrsoftware für Grundschullehrer, die mittels virtueller Realität das richtige Verhalten in Brandsituationen schulen soll. Die Software beinhaltet verschiedene Lernmechanismen wie dynamische Geschichten, Realismus, Bewegungsfreiheit, Level- und Punktsysteme \autocite{DDE_of_VR}. Im Industriellen Kontext existieren bereits Alarm- Management-Systeme, die in VR-Anwendungen für die Fertigung integriert worden sind. Das System dient zur besseren Handhabung von Alarmen und soll somit die Sicherheit von Arbeitnehmern verbessern. Das System kann Alarme automatisch auslösen und zeigt Benutzern visuelle Warnungen an, wenn ein Alarm ausgelöst wird \autocite{Design_of_VR-training}.

\textbf{Aufbauend sind zudem wissenschaftliche Studien} zu VR/AR-Anwendungen vorhanden, die untersuchen, ob und inwieweit sich die Vorbereitung auf Notfälle, die Reaktion während eines Notfalls und die Erholung nach einem Notfall durch den Einsatz von VR und AR-Anwendungen verbessert \autocite{VRandAR}. Einige Produktionsmaschinen und Systeme sind soweit vernetzt, dass sie die Maschinendaten selbständig auswerten und in der Cloud als 3D-Modell visualisieren lassen können. Die Maschine gibt also selber Rückmeldungen über ihren Status und kann ggf. Abweichungen und Fehler selbständig melden \autocite{Mascienenausf_entdecken}.

Ein interessanter Ansatz ist jedoch die Gerätewartung und Diagnose mittels Augmented Reality zu verbessern. Durch die Nutzung des Systems konnten Studierende Datenanalysen und -erfassung für Wartungsanwendungen leichter ausführen. Das Verständnis der Studierenden in diesem Sachbereich wurde durch die Einbindung von AR positiv beeinflusst (\cite{Develop_and_Asses_AR}). Für das ALAARM Projekt sind jedoch \textbf{Echtzeit-Fehlerdiagnosetechnologien (RTFD) für industrielle Prozessüberwachung und Maschinenzustandsüberwachung} besonders interessant. Diese Technologien finden zunehmend Einsatz in der industriellen intelligenten Fertigung und sollten daher unbedingt genauer betrachtet werden. Das im Rahmen des Projekts entworfene System bezieht sich in einem ausgedehnten Maße auf die dort bereits implementierten und entwickelten Technologien.

\newpage




\newpage

%\subsection{Literaturverweise}
%[1] Mystakidis, S., Besharat, J., Papantzikos, G., Christopoulos, A., Stylios, C., Agorgianitis, S. and Tselentis, D. (2022). Design, Development, and Evaluation of a Virtual Reality Serious Game for School Fire Preparedness Training. Education Sciences, 12(4), p.281. doi:https://doi.org/10.3390/educsci12040281. \\

%[2] Matsas, E. and Vosniakos, G. (2015). Design of a virtual reality training system for human–robot collaboration in manufacturing tasks. [online] Available at: https://www.researchgate.net/publication/271837242_Design_of_a_virtual_reality_training_s ystem_for_human-robot_collaboration_in_manufacturing_tasks [Accessed 18 May 2023]. \\


%[3] Zhu, Y. and Li, N. (2021). Virtual and Augmented Reality Technologies for Emergency Management in the Built Environments: A State-of-the-Art Review. Journal of Safety Science and Resilience, Volume 2(March 2021). doi:https://doi.org/10.1016/j.jnlssr.2020.11.004. \\


%[4] Weidle, D. (2023). Maschinenausfälle entdecken bevor sie auftreten. Maschinenausfälle entdecken bevor sie auftreten, 26(5), pp.26–27. doi:https://doi.org/10.1002/citp.202300515. \\

%[5] Shyr, W.-J., Tsai, C.-J., Lin, C.-M. and Liau, H.-M. (2022). Development and Assessment of Augmented Reality Technology for Using in an Equipment Maintenance and Diagnostic System. Sustainability, 14(19), p.12154. doi:https://doi.org/10.3390/su141912154. \\


%[6] Yan, W., Wang, J., Lu, S., Zhou, M. and Peng, X. (2023). A Review of Real-Time Fault Diagnosis Methods for Industrial Smart Manufacturing. Processes, 11(2), pp.369–369. doi:https://doi.org/10.3390/pr11020369.
