\section{Einleitung}
% 1-2 Seiten
Die Industrie 4.0, auch bekannt als die vierte industrielle Revolution, hat eine neue Ära der Automatisierung und Vernetzung in der Fertigungsindustrie eingeläutet. Mit dem Ziel, die Effizienz, Flexibilität und Produktivität zu steigern, werden in der Industrie 4.0 vernetzte Systeme eingesetzt, um eine nahtlose Kommunikation und Koordination zwischen Maschinen, Anlagen und Menschen zu ermöglichen \autocite{reinheimer}.

Die hohe Komplexität dieser Systeme führt für Mitarbeiter wie etwa Anlagenbediener zu neuartigen Herausforderungen. Insbesondere Alarmszenarien stellen im „Industrie 4.0“-Kontext eine bedeutende Herausforderung dar. Aufgrund der Vernetzung und Koordination zwischen den Maschinen(-gruppen) können von einer einzelnen Störung unmittelbar ganze Fertigungsprozesse erfasst werden und eine Kaskade an Unterbrechungen bis zu Folgestörungen sorgen. Um Schäden zu vermeiden und vollumfänglich Sicherheit zu gewährleisten, trägt das Bedienungspersonal die Verantwortung, eine angemessene und zügige Reaktion zu zeigen und zur Beilegung Handlungen zur Entstörung aufzuführen. Das Vorgehen und Verhalten von verantwortlichen Personen bei Eintritt eines Störfalls wurde im Umfeld von Industrie 4.0 Produktionsumgebungen bisher nicht wissenschaftlich untersucht.

Im Rahmen dieses wissenschaftlichen Projekts konzentriert sich auf die Entwicklung von Alarmszenarien im „Industrie 4.0“-Kontext. Hierfür wird eine Dokumentation eines Alarmszenarios in der Form einer Beschreibungskarte und eines BPMN-Prozesses entwickelt. Die Dokumentation wird als Grundlage zur Entwicklung eines Minimum Viable Products („MVP“) herangezogen. Der MVP wird in der Simulationsumgebung des Zentrums Industrie 4.0 am Lehrstuhl für Wirtschaftsinformatik, Prozesse und Systeme der Universität Potsdam eingebettet. Das Alarmszenario bildet damit einen Störfall im simulierten Produktionsumfeld ab, welches Simulationsteilnehmer zu lösen versuchen.

Das Ziel dieses Projektes ist es, die Immersion dieser Alarmszenarien durch den gezielten Einsatz von spezifischen Komponenten zu steigern, um den Stresslevel der Personen zu erhöhen und ihnen ein realitätsnahes Szenario zu bieten. Dies ermöglicht Potentiale zur Forschung von Mensch-Maschinen-Interaktionen im Industrie 4.0 Umfeld, wie etwa der Einschätzung von Reaktionsfähigkeit und Wirksamkeit des individuellen Verhaltens der Teilnehmer in Gefahrensituationen. Darüber hinaus können die entwickelten Alarmszenarien auch zur Entwicklung von Schulungsprogrammen und Übungen genutzt werden, um das Bewusstsein und die Fähigkeiten von Mitarbeitern im Umgang mit potenziellen Gefahrensituationen zu stärken.

In den folgenden Abschnitten werden die methodischen Ansätze, die Definition und Dokumentation eines entwickelten Alarmszenarios und die Einbettung des entwickelten Szenarios in der Simulationsumgebung im Detail vorgestellt. Abschließend wird die Implikationen dieser Forschung diskutieren.
\\

\label{section:name-}
