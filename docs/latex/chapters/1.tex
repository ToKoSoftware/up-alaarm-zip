\section{Einleitung}
% 1-2 Seiten
Die Industrie 4.0, auch bekannt als die vierte industrielle Revolution, hat eine neue Ära der Automatisierung und Vernetzung in der Fertigung eingeläutet. Mit dem Ziel, Effizienz, Flexibilität und Produktivität zu steigern, werden in der Industrie 4.0 vernetzte Systeme eingesetzt, die eine nahtlose Kommunikation und Koordination zwischen Maschinen, Anlagen und Menschen ermöglichen \autocite{reinheimer}.

Die hohe Komplexität dieser Systeme stellt Mitarbeiter, wie beispielsweise Anlagenbediener, vor neue Herausforderungen. Insbesondere Alarmszenarien sind im Kontext der Industrie 4.0 eine bedeutende Problematik. Durch die Vernetzung und Koordination zwischen den Maschinen(-gruppen) kann eine einzelne Störung rasch ganze Fertigungsprozesse beeinträchtigen und zu einer Kaskade von Unterbrechungen und Folgestörungen führen. Um Schäden zu vermeiden und Sicherheit zu gewährleisten, trägt das Bedienungspersonal die Verantwortung, schnell und angemessen zu reagieren und Entstörungsmaßnahmen zu treffen. Das Verhalten verantwortlicher Personen bei einem Störfall in Industrie 4.0 Produktionsumgebungen wurde im wissenschaftlichen Kontext bisher nur wenig untersucht.

Im Rahmen dieses Forschungsprojekts liegt der Fokus auf der Entwicklung von Alarmszenarien im Kontext der „Industrie 4.0“. Dafür wird eine Dokumentation eines Alarmszenarios in Form eines BPMN-Prozesses und Begleitdokumenten erstellt. Diese Dokumentation dient als Basis für die Entwicklung eines Minimum Viable Products (MVP). Der MVP wird in die Simulationsumgebung des Zentrums Industrie 4.0 am Lehrstuhl für Wirtschaftsinformatik, Prozesse und Systeme der Universität Potsdam integriert (nachfolgend \textit{Zentrum Industrie 4.0}). Das Alarmszenario simuliert dabei einen Störfall, den die Simulationsteilnehmer bewältigen müssen.

Ziel des Projekts ist es, die Immersion der Alarmszenarien durch den Einsatz spezifischer Komponenten zu intensivieren, um den Stresslevel der Teilnehmer zu erhöhen und ein realistisches Szenario zu bieten. Dies eröffnet Forschungsmöglichkeiten im Bereich der Mensch-Maschine-Interaktion im Industrie 4.0 Umfeld, wie z.B. die Bewertung von Reaktionsvermögen und Effektivität des individuellen Verhaltens der Teilnehmer in Gefahrensituationen. Zudem können die erstellten Alarmszenarien zur Entwicklung von Schulungsprogrammen und Übungen eingesetzt werden, um das Bewusstsein und die Fähigkeiten der Mitarbeiter im Umgang mit potenziellen Gefahren zu schärfen.

In den nachfolgenden Abschnitten werden die methodischen Ansätze, die Definition und Dokumentation des erstellten Alarmszenarios sowie die Einbindung des Szenarios in die Simulationsumgebung detailliert präsentiert. Abschließend werden die Implikationen dieser Forschung diskutiert.
\\

\label{section:name-}
