\section{Beschreibungskarte}

\ 

\textbf{AGENDA}

\ 

\textbf{3.1 Einleitung und Szenario-Überblick} 
\begin{itemize}
\item
Präsentation des Szenarios und seiner Komponenten
\item
Erklärung des Ablaufs und der Eskalationsstufen
\end{itemize}

\

\textbf{3.2 Sicherheitsbriefing} 
\begin{itemize}
\item
Durchsicht der Gesundheits- und Sicherheitshinweise
\end{itemize}

\

\textbf{3.3 Bedienungsanleitung und Aktivierung der Eskalationsstufe}
\begin{itemize}
\item
Erläuterung der Bedienung der Maschine und des Tablets
\item
Aktivierung der Eskalationsstufe 
\end{itemize}

\

\textbf{3.4 Anzeige der Fehlermeldung und Lösungsansatz }
\begin{itemize}
\item
Erläuterung der Schritte zur Lösung der Fehlermeldungen
\end{itemize}

\

\textbf{3.5 Eskalationsstufe 1}
\begin{itemize}
\item
Durchgehen der Eskalationsstufe 1 und der damit verbundenen Entstörungsvorgang
\item 
Einschreiten der Zufallsvariable
\item
Diskussion über die möglichen Auswirkungen der Zufallsvariable auf den Ausgang des Szenarios
\end{itemize}

\

\textbf{3.6 Eskalationsstufe 2}
\begin{itemize}
\item
Durchgehen der Eskalationsstufe 2 und der damit verbundenen Entstörungsvorgang
\item
Einschreiten der Zufallsvariable
\item
Diskussion über die möglichen Auswirkungen der Zufallsvariable auf den Ausgang des Szenarios
\end{itemize}

 \

\textbf{3.7 Eskalationsstufe 3}
\begin{itemize}
\item
Durchgehen der Eskalationsstufe 2 und der damit verbundenen Entstörungsvorgang
\item
Einschreiten der Zufallsvariable
\item
Diskussion über die möglichen Auswirkungen der Zufallsvariable auf den Ausgang des Szenarios
\end{itemize}

\

\textbf{3.8 Voraussichtliche Zeitangaben des Szenarios zum Abschluss}
\begin{itemize}
\item
Messung der Durchlaufzeit bei optimalen Abläufen
\end{itemize}

\newpage

\textbf{3.1 Einleitung und Szenario-Überblick}

In diesem Szenario wird ein komplexes System aus verschiedenen Komponenten und Prozessen beschrieben, dass in einen Maschinen-Simulationsraum implementiert wird. Die Szenariokomponenten bestehen aus Teilnehmer, Tablet, Subwoofer/Lautsprecher, Wärmelampe/Licht und Vaporizer. Diese Komponenten sind in einer Umgebung platziert, in der es Anlagen zum Bedienen gibt, die alle durch ein Förderband miteinander verbunden sind. Das System ist so konzipiert, dass es drei Eskalationsstufen gibt, die nacheinander aktiviert werden können. Jede Eskalationsstufe beinhaltet einen Entstörungsvorgang, der vom Teilnehmer gelöst werden muss. Dabei werden verschiedene Szenariokomponenten aktiviert, um den Prozess zu unterstützen und zu simulieren. Teilnehmer des Szenarios, die bei einem Enstörungsvorgang versagen, werden automatisch zur nächsten Eskalationsstufe befördert, bis sie maximal die Eskalationsstufe 3 erreichen. Während eines Enstörungsvorgangs steht den Teilnehmern dabei nur ein Versuch zur Verfügung. Allerdings haben erfolgreiche Szenarioteilnehmer die Möglichkeit, nach Abschluss einer beliebigen Eskalationsstufe das gesamte Szenario vorzeitig zu beenden. Pro Durchgang führt nur ein Szenarioteilnehmer das Szenario durch. Es gibt einen Betreuer, der diese Vorgänge beobachtet und bei Notfällen einschreitet, jedoch ist er nicht für die gesamten Handlungen im Szenarioprozess eingeplant.

\

\underline{Einleitung zum Szenario:} \\
Der Teilnehmer aktiviert die Anlage mit einem Startknopf auf dem Tablet. (Zeit: ca. 1 Sekunde)
Es wird das Datenpaket für den Szenarioteilnehmer ausgelesen, woraufhin Anweisungen wie Gesundheitswarnungen oder weitere Verfahren angezeigt werden, die der Teilnehmer lesen soll: (Zeit: ca. 2 Minute)

\

\textbf{3.2 Sicherheitsbriefing}

\emph{Bitte nehmen Sie sich die Zeit, diese Anweisungen sorgfältig zu lesen und zu verstehen, bevor Sie mit dem Betrieb der Maschine fortfahren. Ihre Sicherheit hat oberste Priorität.}

\newpage

\emph{WICHTIG: GESUNDHEITSWARNUNG \\
Bitte stellen Sie sicher, dass Sie die geeignete Schutzkleidung tragen, bevor Sie mit dem Betrieb dieser Maschine beginnen. Dies beinhaltet Schutzhandschuhe, Schutzbrille und Atemschutzmaske. Bitte entnehmen Sie die Schutzkleidung, die rechts von Ihnen lieg. 
Es besteht die Möglichkeit, dass während des Szenarios Rauch oder andere intensiv riechende Substanzen verwendet werden.
Dieses Szenario beinhaltet flackernde Lichter, die bei einigen Personen mit Epilepsie zu Anfällen führen können. 
Bitte beachten Sie, dass Ihre Gesundheit und Sicherheit oberste Priorität haben. Falls Sie Bedenken haben oder gesundheitliche Einschränkungen haben, die sich auf Ihre Teilnahme an diesem Szenario auswirken könnten, empfehlen wir Ihnen, vorab Rücksprache mit einem Arzt zu halten, um sicherzustellen, dass Sie angemessen geschützt sind und das Szenario ohne Risiken für Ihre Gesundheit durchführen können.}

\

\textbf{3.3 Bedienungsanleitung und Aktivierung der Eskalationsstufe}
\

\emph{BEDIENUNGSANLEITUNG \\
1.	Überprüfen Sie, ob alle Sicherheitsvorkehrungen getroffen wurden und die Maschine bereit für den Betrieb ist. \\
2.	Drücken Sie den grünen Startknopf auf der rechten Seite des Bedienfelds, um den Maschinenprozess zu starten. \\
3.	Überwachen Sie den Prozess auf dem Bildschirm. Achten Sie auf alle Warnungen oder Fehlermeldungen, die angezeigt werden könnten. \\
4.	Im Falle einer Fehlermeldung, drücken Sie den roten Stopp-Knopf und befolgen Sie die Anweisungen auf dem Bildschirm. \\
5.	Wenn der Prozess abgeschlossen ist, drücken Sie den blauen Reset-Knopf, um die Maschine auf ihren Ausgangszustand zurückzusetzen.}

\

\textbf{\underline{Start:}}

\

\underline{Einleitung der Eskalationsstufe 1:}

Die Anlage fängt an zu laufen. (Zeit: ca. 8 Sekunden) 
Gleichzeitig werden in der ersten Stufe die Szenariokomponenten aktiviert, wie der Subwoofer/Lautsprecher, der einen langsamen pulsierenden Ton erzeugt, sowie die Wärmelampe/Licht (Andon-Signalleuchte), die an der jeweiligen Maschine aktiviert wird und kontinuierlich leuchtet. (Zeit: ca. 1 Sekunde)
Die Anlage ruckelt und stoppt ihren Arbeitsfluss. (Zeit: ca. 5 Sekunden)

\

\textbf{3.4 Anzeige der Fehlermeldung und Lösungsansatz}

Auf dem Tablet erscheint ein Fehlercode mit dazugehörigen Maschinen- und Fehlerdetails, die der Szenarioteilnehmer liest: (Zeit: ca. 40 Sekunden)

\

\emph{FEHLERCODE: E101\\
BETROFFENE MASCHINE: Maschine 3\\
FEHLERDETAILS: Überhitzung}

\

\emph{Die Sensoren in Maschine 3 haben eine übermäßige Wärmeentwicklung festgestellt, die über den sicheren Betriebsbereich hinausgeht. Dies kann zu Schäden an der Maschine und möglicherweise zu einem unsicheren Arbeitsumfeld führen.}

\emph{Stoppen Sie Maschine 3 sofort, um weitere Schäden zu vermeiden, und lösen Sie das Rätsel während des Entstörungsvorgangs!}

Hinlaufen zu dieser Maschine. (Zeit: ca. 15 Sekunden)

\

\textbf{3.5 Eskalationsstufe 1}

Auf dem Tablet erscheint ein QR-Code, der gescannt werden muss, um einen Entstörungsvorgang in 40 Sekunden bzw. das Problem zu lösen wie: (Zeit: ca. 45 Sekunden)

\emph{(!) Lösen Sie das folgende Rätsel, um den Fehler zu beheben!}

\

\underline{\emph{Nummernspiel}} \\
\emph{Auf dem Tablet wird eine Zahlenfolge angezeigt:} \\

\

\emph{(!) Lösen Sie das folgende Rätsel, um den Fehler zu beheben!} \\

\emph{Auf dem Tablet wird eine zufällige Zahlenfolge angezeigt:} \\
\emph{3, 5, 8, 10, 9, 2, 7, 4, 1, 6} \\

\
\emph{Ihre Aufgabe ist es, die Zahlen in aufsteigender Reihenfolge anzuklicken. Sie haben 10 Sekunden Zeit, um das Rätsel zu lösen.}

\

Es folgt eine Verzweigung, die eine Zufallsvariable beinhaltet:

{Nach erfolgreichem Abschluss des Entstörungsvorgangs wird auf dem Tablet eine Benachrichtigung angezeigt, die den Abschluss des Vorgangs bestätigt. Um die Wahrscheinlichkeit von 33\% zu simulieren, wird eine digitale Würfelfunktion verwendet, die die Zahlen 1, 2 und 3 generiert und damit drei mögliche Zustände repräsentiert. Nach dem Entstörungsvorgang wird die digitale Würfelfunktion ausgeführt. Wenn die gewürfelte Zahl eine 1 oder 2 ist (entspricht einer Wahrscheinlichkeit von 2/3), wird auf dem Tablet angezeigt, dass das Problem als gelöst erfasst wurde. Falls die gewürfelte Zahl eine 3 ist (entspricht einer Wahrscheinlichkeit von 1/3), wird auf dem Tablet angezeigt, dass das Problem nicht vollständig gelöst wurde. \\

\begin{itemize}
\item
Sollte der Entstörungsvorgang innerhalb der vorgegebenen Zeit nicht erfolgreich abgeschlossen werden oder eine falsche Eingabe getätigt werden, erfolgt der Übergang zur zweiten Eskalationsstufe automatisch. (Keine genaue Zeitangabe)
\item
Ist der Entstörungsvorgang erfolgreich gelöst, folgt die Zufallsvariable, dabei wird bei einer 1/3 Wahrscheinlichkeit das Problem als nicht gelöst erfasst. Dann erfolgt ebenfalls der Übergang zur zweiten Eskalationsstufe automatisch. (Keine genaue Zeitangabe)
\item
Wenn der Entstörungsvorgang erfolgreich gelöst ist, folgt die Zufallsvariable, bei der mit einer Wahrscheinlichkeit von 2/3 das Problem als gelöst erfasst wird. Wenn dieses Szenario eintrifft, werden alle Szenariokomponenten ausgeschaltet und das Szenario für den Teilnehmer ist erfolgreich beendet.  (Zeit: ca. 3 Sekunden)
\end{itemize}


\newpage 

\textbf{3.6 Eskalationsstufe 2}

Simultan werden die Szenariokompontenten Subwoofer/Lautprecher mit einem mittelschnell schlagenden Ton aktiviert. Der Vaporizer stößt derweil Brandgeruch aus. Die LED-Leuchte flackert bei der Maschine blau. Diese bleiben durchgehend in der zweiten Eskalationsstufe aktiv. (Zeit: ca. 5 Sekunde) \\
Der Szenarioteilnehmer soll sich selbstständig zur nächsten Maschine bewegen, was durch eine Andon-Signalleuchte an der Maschine signalisiert wird. (Zeit: ca. 15 Sekunden) \\
Erneut muss ein Entstörungsvorgang durchgeführt werden: (Zeit: ca. 40 Sekunden) 

\

\underline{\emph{Kalibrierung}} \\
\emph{(!) Lösen Sie das folgende Rätsel, um den Fehler zu beheben!}\\

Es gibt ein Problem mit der Kalibrierung der Maschine, was zu schwerwiegenden Fehlfunktionen führt. Es ist entscheidend, dass die Eingaben der Maschine korrekt kalibriert sind, um Fehlerquellen zu eliminieren. Um eine korrekte Kalibrierung zu gewährleisten, klicken Sie so schnell wie möglich auf den Bildschirm, sobald er grün wird. \\
          
Der Bildschirm startet im schwarzen Zustand. Wenn Sie länger als 0,6 Sekunden brauchen, um eines der beiden grünen Kacheln anzuklicken, kann die Maschine nicht kalibriert werden. Es sind 6 Kalibrierungseingaben erforderlich, um die Maschine erfolgreich zu kalibrieren.

$\Box\hspace{85pt}\\\Box$

\

Danach erfolgt eine weitere Verzweigung, bei der die Zufallsvariable aktiviert wird. \\

\

\textbf{3.7 Eskalationsstufe 3} \\

Gleichzeitig werden die Szenariokomponenten Subwoofer/Lautsprecher mit einem schnell schlagenden Ton, der Vaporizer mit einem intensiveren Geruch als zuvor und die Wärmelampe aktiviert. Die Andon-leuchte flackert rot. Diese bleiben durchgehend in der dritten Eskalationsstufe aktiv. (Zeit: ca. 5 Sekunde) \\
Der Szenarioteilnehmer soll zur nächsten Maschine, welches durch eine Andon-Signalleuchte signalisiert wird, hinbewegen. (Zeit: ca. 15 Sekunden) \\
Erneut muss ein Entstörungsvorgang durchgeführt werden: (Zeit: ca. 40 Sekunden) 

\

\underline{\emph{Algorithmus}} \\
\emph{Auf dem Tablet wird eine Zahlenfolge angezeigt:} \\

\

\emph{(!) Lösen Sie das folgende Rätsel, um den Fehler zu beheben!} \\

\
\emph{Der Algorithmus der Maschine scheint ein Problem zu haben. \\
Um die Maschine zu reparieren, müssen Sie die Zahlen 1 bis 10 in aufsteigender Reihenfolge anklicken.
Nach 5 und 10 Sekunden werden die Zahlen durcheinandergebracht. \\
Sobald Sie beginnen, haben Sie 15 Sekunden Zeit.}

\

\emph{Die Zahlen werden auf dem Bildschirm in einer zufälligen Reihenfolge verstreut angezeigt:} \\
\emph{3\hspace{15pt} 5\hspace{42pt} 8\hspace{58pt} 10\hspace{75pt} \\9\hspace{25pt} 2\hspace{37pt} 7\hspace{83pt} \\4\hspace{39pt} 1\hspace{66pt} 6\hspace{81pt}} \\

\
Danach erfolgt eine weitere Verzweigung, bei der die Zufallsvariable aktiviert wird.

\

\textbf{3.8 Voraussichtliche Zeitangaben des Szenarios}\\
Bitte beachten Sie, dass dies geschätzte Zeiten sind und je nach individueller Durchführung und Reaktionsgeschwindigkeit des Teilnehmers leicht variieren können. Diese Zeitangaben stehen daher unter Vorbehalt und dienen lediglich als grobe Orientierung. Es ist möglich, dass die tatsächliche Dauer für jeden Teilnehmer unterschiedlich ist.

\

\underline{Benötigte Zeit für Eskalationsstufe 1} \\
Lesen der Anweisungen: ca. 2 Minuten \\
Aktivierung der Anlage: ca. 8 Sekunden \\
Unterbrechung der Anlage ca. 5 Sekunden \\
Anzeige der Fehlermeldung: ca. 40 Sekunden \\
Hinlaufen zur Maschine: ca. 15 Sekunden \\
Entstörungsvorgang in der ersten Stufe: ca. 10 Sekunden \\
Einschreiten der Zufallsvariable: ca. 3 Sekunden \\
Geschätzte Gesamtzeit für den Abschluss der ersten Stufe: ca. 3 1/2 Minuten \\

\

\underline{Benötigte Zeit für Eskalationsstufe 2} \\
Zeit für die erste Stufe: ca. 3 1/2 Minuten \\
Hinlaufen zur Maschine: ca. 15 Sekunden \\
Entstörungsvorgang in der zweiten Stufe: ca. 40 Sekunden \\
Einschreiten der Zufallsvariable: ca. 3 Sekunden \\
Geschätzte Gesamtzeit für den Abschluss der ersten und der zweiten Stufe: ca. 4 1/2 Minuten \\

\

\underline{Benötigte Zeit für Eskalationsstufe 3 / Max. Zeit bei Erfolglosigkeit} \\
Zeit für die zweite Stufe: ca. 4 1/2 Minuten \\
Hinlaufen zur Maschine: ca. 15 Sekunden \\
Entstörungsvorgang in der dritten Stufe: ca. 15 Sekunden \\
Einschreiten der Zufallsvariable: ca. 3 Sekunden \\
Geschätzte Gesamtzeit für den Abschluss der ersten, zweiten und der dritten Stufe: ca. 5 Minuten \\




