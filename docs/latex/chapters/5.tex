\section{Technische Dokumentation}

Im folgendem Abschnitt wird auf die Implementierung des Alarmszenarios und dazugehörige Überlegungen eingegangen. Zu Beginn wird der gewählte Teckstack dargelegt und durch Ausführungen zur Systemarchitektur ergänzt. 
Daran schließt ein Abschnitt zur Erklärungen der Konfigurationen von Alarmszenarien als auch eingesetzten APIs an.

\subsection{Techstack}

Die Auswahl des Teckstacks zur Implementierung von ALAARM-ZIP hängt zentral von einer Vielzahl von Vorbedingungen ab.

Bezüglich der verwendeten Programmiersprache wurde sich für den Einsatz von TypeScript entschieden. Neben der Eignung für Webentwicklung

Durch die Vorbedingungen des Einsatzes von Tablets durch den Administrator zur Konfiguration und Steuerung, der ähnlichen Anwendung durch Simulationsteilnehmer zur Lösung von Rätseln und der Bereitstellung von QR-Codes auf Bildschirmen der simulierten Maschinen

In Anbetracht der Tatsache, dass einerseits Administratoren mithilfe von Tablets Szenarien konfigurieren und steuern können und Teilnehmer der Übungssimulation

\subsection{Systemarchitektur}

\subsection{Konfiguration von Alarmszenarien}

\subsection{APIs}