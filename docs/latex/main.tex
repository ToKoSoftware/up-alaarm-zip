\documentclass[12pt]{article}
\usepackage[a4paper, left=3cm, right=3cm, top=3cm, bottom=3cm]{geometry}
\usepackage[onehalfspacing]{setspace}
\usepackage{microtype}
\usepackage[T1]{fontenc}
\usepackage[utf8]{inputenc}
\usepackage[english, ngerman]{babel}
\usepackage{parskip} 
\usepackage{abstract}
\usepackage{float}
\usepackage{amsmath}
\usepackage{amstext}
\usepackage{varioref}
\usepackage[unicode=true,pdfusetitle,
 bookmarks=true,bookmarksnumbered=false,bookmarksopen=false, breaklinks=false,pdfborder={0 0 0},backref=false,colorlinks=false] {hyperref}
\usepackage{graphicx}
\usepackage{setspace}
\usepackage{lipsum}
\usepackage{cleveref}
\usepackage{icomma}
\usepackage{tabu,booktabs}
\usepackage{tcolorbox}
\usepackage{csquotes}
\usepackage{enumitem}
\usepackage{xspace}

\usepackage{tabularx}
\usepackage{pdflscape}
\usepackage{rotating}
\usepackage{booktabs}
\usepackage{array}
\newcolumntype{P}[1]{>{\centering\arraybackslash}p{#1}}
\renewcommand{\arraystretch}{1.1}
\usepackage{listings}
\usepackage{acronym}
\usepackage{minted}
\usepackage{longtable}
\usepackage{amssymb}
\usepackage{arabtex}
\usepackage{enumitem}
\setlist[itemize]{topsep=-0.2cm}
\usepackage[format=plain,
            font={it, small}]{caption}
\captionsetup{justification=raggedright,singlelinecheck=false}
\usepackage{threeparttable}
\usepackage{lastpage}
\usepackage{fancyhdr}
\pagestyle{fancy}
\fancyhead[R]{\nouppercase{\leftmark}}
\fancyhead[L]{}
\renewcommand{\headruleskip}{3pt}
\usepackage{afterpage}
\usepackage{tablefootnote}
\usepackage[hang,flushmargin]{footmisc}
%\usepackage{showframe}
% % % % % % % % % % % % % % % % % % % % %
%Schriften - nur eine auskommentieren
	%Latin Modern - Standard
\usepackage{lmodern}
	%Times Roman für strenge Dozenten
		%\usepackage{times}
	%Garamond - sieht toll aus, aber eher exotisch
		%\usepackage[cmintegrals,cmbraces]{newtxmath}\usepackage{ebgaramond-maths} %\usepackage{helvet}

%Hurenkinder, Schusterjungen
\widowpenalties=3 10000 10000 150

\newcommand*\MeasuredFigureLabel[1]{
    \addtocounter{figure}{-1}
    \phantomcaption
    \label{#1}
}

% % % % % % % % % % % % % % % % % % % % %
%LITERATUR
%Fußnote _ODER_ Amerikanisch auskommentieren

	%Amerikanische In-Text-Zitierweise
\usepackage[english]{babel}
\usepackage{csquotes}% Recommended
\usepackage[style=authoryear-ibid,backend=biber]{biblatex}


	%Deutsches Fußnotensystem
		%\usepackage[backend=biber,style=authortitle-dw,firstfull,maxcitenames=1]{biblatex}\DeclareFieldFormat{title}{\mkbibemph{#1}}\DeclareFieldFormat{citetitle}{\mkbibemph{#1}}\renewcommand{\cite}{\footcite}

%Ressource
\bibliography{bibtex.bib}
% for et. al. in italic
\usepackage{xpatch}
\xpatchbibmacro{name:andothers}{%
  \bibstring{andothers}%
}{%
  \bibstring[\emph]{andothers}%
}{}{}
% For using '&' insead of 'und' when \cite is used -> for text use \textcite
\DeclareDelimFormat[bib,cite]{finalnamedelim}{%
  \ifnumgreater{\value{liststop}}{2}{\finalandcomma}{}%
  \addspace\&\space}

% % % % % % % % % % % % % % % % % % % % %
\makeatletter


\begin{document}

%%Hier alle Daten einfügen
\newcommand{\autornameeins}{Marwin Ahnfeldt, B.Sc. ($819026$)}
\newcommand{\autornamezwei}{Tomas Kostadinov, B.Sc. ($820862$)}
\newcommand{\autornamedrei}{Laurenz Gilbert ($808291$)}
\newcommand{\autornamevier}{Jan-Hendrik Höltkemeier ($815258$)}
\newcommand{\autornamefuenf}{Victoria Schock ($797109$)}
\newcommand{\autornamesechs}{Fahian Arshad Bhuiyan ($806839$)}
\newcommand{\autornamesieben}{Eik Saathoff ($809906$)}
\newcommand{\fak}{Wirtschafts- und Sozialwissenschaftliche Fakultät}
\newcommand{\lehrstuhl}{Wirtschaftsinformatik, insb. Prozesse und Systeme}
\newcommand{\lv}{Analyse von Geschäftsprozessen \& Konzeption von IT Systemen (A\&K)}
\newcommand{\zeitraum}{Sommersemester 2023}
\newcommand{\lp}{M.Sc. Marcel Panzer}
\newcommand{\arbtyp}{Schaffung eines hochimmersiven (Lern-)Szenario für hybride Simulationsumgebungen}
\newcommand{\titel}{ALAARM!}

\title{\titel}
\author{\autornameeins \autornamezwei \autornamedrei}

\newcommand{\TBA}{\texttt{\color{red}Todo}}

\begin{titlepage}
\vspace*{4.5cm}
\centering
\includegraphics[width=2.5cm]{res/uni_potsdam_logo.pdf}
\vfill
\begin{tabu} to \textwidth {X[c]}
	\huge{\textbf{{\textsc{\titel}}}}\\\\
	\large{\textsc{\arbtyp}}\\
\end{tabu}
\\[1cm]
\centering{
\autornameeins\\
\autornamesechs\\
\autornamedrei\\
\autornamevier\\
\autornamezwei\\
\autornamesieben\\
\autornamefuenf\\
}
\vfill
\begin{table}[h]
    \small
    \begin{tabular}{ll}
        Fakultät: & \fak \\
        Lehrstuhl: & \lehrstuhl \\
        Lehrveranstaltung: & \lv \\
        Bearbeitungszeitraum: & \zeitraum \\
        Dozent: & \lp
    \end{tabular}
\end{table}

\end{titlepage}

\renewcommand{\thepage}{\Roman{page}}
\setcounter{page}{1}
{\small\tableofcontents}
\cleardoublepage

\pagebreak{}
\selectlanguage{english} 
\begin{abstract}
Die vorliegende Dokumentation bietet einen umfassenden Einblick in die Konzeption und Umsetzung eines hochimmersiven Lernszenarios für eine hybride Simulationsumgebung. Sie behandelt ein breites Spektrum an Aspekten, darunter eine gründliche Recherche zu existierenden kommerziellen Lösungen, Übungsszenarien, Standardabläufen und wissenschaftlichen Studien. Es wird hervorgehoben, dass bereits immersive Softwarelösungen für das Training und die Vorbereitung auf kritische Situationen existieren. Ein Beispiel ist eine Anwendung, die Grundschullehrern mittels virtueller Realität das richtige Verhalten in Brandsituationen vermittelt. Darüber hinaus gibt es industrielle Alarm-Management-Systeme, die in VR-Anwendungen für die Fertigung integriert sind. Diese Systeme zielen darauf ab, die Handhabung von Alarmen zu verbessern und somit die Sicherheit der Arbeitnehmer zu erhöhen. Die Konzeption des finalen Szenarios folgte einem mehrstufigen Ansatz. Zunächst wurde ein individuelles Brainstorming durchgeführt, bei dem verschiedene Komponenten und Abläufe der Szenarien diskutiert wurden. Verschiedene Ideen wurden erörtert, darunter die Verwendung von Alarmsignalen in Form von Bass oder Sirene, die Verwendung von Brandgeruch als Warnsignal und den Einsatz von visuellen Effekten mit flackerndem Licht. Ein weiterer, ebenso wichtiger Bestandteil der Dokumentation ist die "Beschreibungskarte", die den Ablauf des Szenarios und seine Eskalationsstufen detailliert darstellt. Das Szenario ist komplex aufgebaut und umfasst eine Vielzahl von Komponenten, darunter ein Teilnehmer, ein Tablet und verschiedene physische Elemente wie einen Subwoofer/Lautsprecher, eine Wärmelampe/Licht und einen Vaporizer. Es ist in drei Eskalationsstufen strukturiert, die nacheinander aktiviert werden. Jede Eskalationsstufe beinhaltet einen spezifischen Entstörungsvorgang, den der Teilnehmer lösen muss. Es werden auch geschätzte Zeitangaben für die Durchführung des Szenarios gegeben.
\end{abstract}

\pagebreak

\selectlanguage{ngerman} 


\onehalfspacing

%\section{Einleitung}
% 1-2 Seiten
Die Industrie 4.0, auch bekannt als die vierte industrielle Revolution, hat eine neue Ära der Automatisierung und Vernetzung in der Fertigung eingeläutet. Mit dem Ziel, Effizienz, Flexibilität und Produktivität zu steigern, werden in der Industrie 4.0 vernetzte Systeme eingesetzt, die eine nahtlose Kommunikation und Koordination zwischen Maschinen, Anlagen und Menschen ermöglichen \autocite{reinheimer}.

Die hohe Komplexität dieser Systeme stellt Mitarbeiter, wie beispielsweise Anlagenbediener, vor neue Herausforderungen. Insbesondere Alarmszenarien sind im Kontext der Industrie 4.0 eine bedeutende Problematik. Durch die Vernetzung und Koordination zwischen den Maschinen(-gruppen) kann eine einzelne Störung rasch ganze Fertigungsprozesse beeinträchtigen und zu einer Kaskade von Unterbrechungen und Folgestörungen führen. Um Schäden zu vermeiden und Sicherheit zu gewährleisten, trägt das Bedienungspersonal die Verantwortung, schnell und angemessen zu reagieren und Entstörungsmaßnahmen zu treffen. Das Verhalten verantwortlicher Personen bei einem Störfall in Industrie 4.0 Produktionsumgebungen wurde im wissenschaftlichen Kontext bisher nur wenig untersucht.

Im Rahmen dieses Forschungsprojekts liegt der Fokus auf der Entwicklung von Alarmszenarien im Kontext der „Industrie 4.0“. Dafür wird eine Dokumentation eines Alarmszenarios in Form eines BPMN-Prozesses und Begleitdokumenten erstellt. Diese Dokumentation dient als Basis für die Entwicklung eines Minimum Viable Products (MVP). Der MVP wird in die Simulationsumgebung des Zentrums Industrie 4.0 am Lehrstuhl für Wirtschaftsinformatik, Prozesse und Systeme der Universität Potsdam integriert (nachfolgend \textit{Zentrum Industrie 4.0}). Das Alarmszenario simuliert dabei einen Störfall, den die Simulationsteilnehmer bewältigen müssen.

Ziel des Projekts ist es, die Immersion der Alarmszenarien durch den Einsatz spezifischer Komponenten zu intensivieren, um den Stresslevel der Teilnehmer zu erhöhen und ein realistisches Szenario zu bieten. Dies eröffnet Forschungsmöglichkeiten im Bereich der Mensch-Maschine-Interaktion im Industrie 4.0 Umfeld, wie z.B. die Bewertung von Reaktionsvermögen und Effektivität des individuellen Verhaltens der Teilnehmer in Gefahrensituationen. Zudem können die erstellten Alarmszenarien zur Entwicklung von Schulungsprogrammen und Übungen eingesetzt werden, um das Bewusstsein und die Fähigkeiten der Mitarbeiter im Umgang mit potenziellen Gefahren zu schärfen.

In den nachfolgenden Abschnitten werden die methodischen Ansätze, die Definition und Dokumentation des erstellten Alarmszenarios sowie die Einbindung des Szenarios in die Simulationsumgebung detailliert präsentiert. Abschließend werden die Implikationen dieser Forschung diskutiert.
\\

\label{section:name-}
 \clearpage
\section{Einleitung}
% 1-2 Seiten
Die Industrie 4.0, auch bekannt als die vierte industrielle Revolution, hat eine neue Ära der Automatisierung und Vernetzung in der Fertigung eingeläutet. Mit dem Ziel, Effizienz, Flexibilität und Produktivität zu steigern, werden in der Industrie 4.0 vernetzte Systeme eingesetzt, die eine nahtlose Kommunikation und Koordination zwischen Maschinen, Anlagen und Menschen ermöglichen \autocite{reinheimer}.

Die hohe Komplexität dieser Systeme stellt Mitarbeiter, wie beispielsweise Anlagenbediener, vor neue Herausforderungen. Insbesondere Alarmszenarien sind im Kontext der Industrie 4.0 eine bedeutende Problematik. Durch die Vernetzung und Koordination zwischen den Maschinen(-gruppen) kann eine einzelne Störung rasch ganze Fertigungsprozesse beeinträchtigen und zu einer Kaskade von Unterbrechungen und Folgestörungen führen. Um Schäden zu vermeiden und Sicherheit zu gewährleisten, trägt das Bedienungspersonal die Verantwortung, schnell und angemessen zu reagieren und Entstörungsmaßnahmen zu treffen. Das Verhalten verantwortlicher Personen bei einem Störfall in Industrie 4.0 Produktionsumgebungen wurde im wissenschaftlichen Kontext bisher nur wenig untersucht.

Im Rahmen dieses Forschungsprojekts liegt der Fokus auf der Entwicklung von Alarmszenarien im Kontext der „Industrie 4.0“. Dafür wird eine Dokumentation eines Alarmszenarios in Form eines BPMN-Prozesses und Begleitdokumenten erstellt. Diese Dokumentation dient als Basis für die Entwicklung eines Minimum Viable Products (MVP). Der MVP wird in die Simulationsumgebung des Zentrums Industrie 4.0 am Lehrstuhl für Wirtschaftsinformatik, Prozesse und Systeme der Universität Potsdam integriert (nachfolgend \textit{Zentrum Industrie 4.0}). Das Alarmszenario simuliert dabei einen Störfall, den die Simulationsteilnehmer bewältigen müssen.

Ziel des Projekts ist es, die Immersion der Alarmszenarien durch den Einsatz spezifischer Komponenten zu intensivieren, um den Stresslevel der Teilnehmer zu erhöhen und ein realistisches Szenario zu bieten. Dies eröffnet Forschungsmöglichkeiten im Bereich der Mensch-Maschine-Interaktion im Industrie 4.0 Umfeld, wie z.B. die Bewertung von Reaktionsvermögen und Effektivität des individuellen Verhaltens der Teilnehmer in Gefahrensituationen. Zudem können die erstellten Alarmszenarien zur Entwicklung von Schulungsprogrammen und Übungen eingesetzt werden, um das Bewusstsein und die Fähigkeiten der Mitarbeiter im Umgang mit potenziellen Gefahren zu schärfen.

In den nachfolgenden Abschnitten werden die methodischen Ansätze, die Definition und Dokumentation des erstellten Alarmszenarios sowie die Einbindung des Szenarios in die Simulationsumgebung detailliert präsentiert. Abschließend werden die Implikationen dieser Forschung diskutiert.
\\

\label{section:name-}
 \clearpage
\section{Hintergrund}

Im folgenden Kapitel werden verwandte Themenbereiche zu Alarmszenarien und Alarmsimulationen vorgestellt. Zunächst werden die Rechercheergebnisse zu kommerziellen Lösungen für Alarmszenarien (Kapitel \ref{chap:commercial_research}) behandelt. Anschließend wird ein Überblick über Übungsszenarien (Kapitel \ref{chap:standart_research}) gegeben, gefolgt von einer Übersicht zur wissenschaftlichen Fachliteratur  (Kapitel \ref{chap:scientific_research}) zum gleichen Themenbereich.

\subsection{Kommerzielle Lösungen}
\label{chap:commercial_research}
Im Branchenumfeld des Maschinenbaus existiert eine Vielzahl von \textbf{kommerziellen Alarmsystemlösungen}. Im Folgenden sind einige dieser Angebote aufgeführt, die durch zusätzliche Hardware Echtzeitüberwachung ermöglichen:

\begin{description}
    \item [\textbf{HGP-Eberle:}] 
    Ein Cloud-basiertes Alarmierungssystem mit permanenter Überwachung durch eine direkt an die Maschine angeschlossene Alarm-Box. Bei Störungen benachrichtigt das System zuständiges Personal mittels einer App. (\cite{HGP-Eberle})
   
   \item [\textbf{Ixon-Cloud:}] 
   Webbasiertes Alarmierungssystem, welches Empfänger weltweit alarmiert. Maschinen werden kontinuierlich durch ein Edge-Gerät überwacht, wodurch Datenanalysen für Dashboards und Reports ermöglicht werden. (\cite{Ixon-Cloud})
   
   \item [\textbf{Mobeye:}] 
   Ein Gerät überwacht die Stromversorgung der Maschine. Im Falle einer Störung wird eine Alarmbenachrichtigung per App oder E-Mail versendet. Dabei können Nutzer zwischen verschiedenen Benachrichtigungsabläufen wählen. (\cite{Mobeye})
   
   \item [\textbf{Totmannschalter:}] 
    Ein Gerät zur Überwachung von allein arbeitenden Personen. Bei Handlungsunfähigkeit oder durch manuellen Auslöser wird ein Notruf versendet, wobei die Person via GPS geortet wird. (\cite{totmann})
\end{description}
 
Des Weiteren unterscheiden sich folgende kommerzielle Benachrichtigungssysteme:

\begin{description}
   \item [\textbf{Workerbase:}] Diese App zeigt Maschinenausfälle als Warnung auf Endgeräten ausgewählter Mitarbeiter an. Dabei sind unterschiedliche Alarmierungs- und Benachrichtigungsfunktionen verfügbar. Die im Workerbase-System gesammelten Daten können über die App analysiert werden.(\cite{Workerbase})
   
   \item [\textbf{safeREACH:}] Durch diese App können alle betroffenen Personen in verschiedenen Notfallsituationen kontaktiert werden. Das Auslösen eines Alarms erfolgt manuell oder automatisch durch Drittsysteme. Nach der Alarmierung stehen Tools für angemessene Maßnahmen bereit. Alle Aktivitäten werden automatisch protokolliert. (\cite{safeREACH})
   
   \item [\textbf{Vedosign:}] Per Knopfdruck ausgelöste Alarme senden Textnachrichten an nahegelegene mobile Geräte, um den passenden Mitarbeiter für den Alarm zu benachrichtigen. (\cite{vedosign})
   
   \item [\textbf{Everbridge:}] Diese Software versendet Nachrichten an spezifische Personengruppen oder Bevölkerungsteile in Gefahrenzonen. (\cite{everbridge})

   \item [\textbf{Gisbo:}] Diese Alarmierungssoftware unterstützt das Krisenmanagement in Gefahrensituationen. Verschiedene Alarmarten, sowohl akustisch als auch optisch, können über die bestehende IT-Infrastruktur weitergeleitet werden.(\cite{Gisbo})

   \item [\textbf{WEKA:}] Elektroakustische Notfallsysteme gemäß DIN EN 50849 warnen Personen vor Ort und fordern zur Selbstrettung auf. Unterschiedliche Signaltöne und gespeicherte Alarmtexte werden durch Sirenen übermittelt. (\cite{WEKA})
   
   \item [\textbf{Videc:}] Dieses System ermöglicht sofortige und zielgerichtete Benachrichtigungen per E-Mail, SMS, Sprachnachricht, Audio, Messenger oder Pager an beteiligte Empfänger. (\cite{VIDEC})

   \item[ISA Telematics:] Das Angebot dieses Anbieters umfasst die Sicherheits-App ''iTProtection'', worüber Notrufe ausgelöst und Personen geortet werden können, und die Telematikplattform ''iTelematics HL'', auf der Alarmmeldungen verwaltet werden. Nutzer erhalten akustische Meldungen und können auf Alarmpläne oder medizinische Informationen zugreifen. (\cite{ISA_Telematics})

\end{description} \ 

\newpage
\subsection{Simulation von Alarmszenarien}
\label{chap:standart_research}
In verschiedenen Zusammenhängen wurden Richtlinien, Handbücher und Empfehlungen zum \textbf{Inhalt und zur Durchführung von Sicherheitsübungen} festgehalten. \\
Grundsätzlich wird dabei zwischen zwei Arten von Übungen unterschieden:
\begin{itemize}
    \item \textbf{Simulations- oder Stabsrahmenübungen: } Üben von Führungsfunktionen und Aufgaben von Einsatzkräften in geschlossenen Räumen (→ strategische Entscheidungen). Diese Übungen sind mit weniger Aufwand verbunden und dadurch realitätsferner.   
    \item \textbf{Vollübungen: }Reales Handeln der Übungsteilnehmer. Das gesamte Einsatzgeschehen wird innerhalb der Übung möglichst realitätsnah abgebildet.
\end{itemize}\ 


Der „Leitfaden für strategische Krisenmanagement-Übungen“ vom \textbf{Bundesamt für Bevölkerungsschutz und Katastrophenhilfe} (\cite{strat_KrisenMGMT}) soll der effektiven Vorbereitung auf eine Krisensituation dienen. Angewendet werden die Inhalte unter anderem bei der länder- und ressortübergreifende Krisenmanagementübung (LÜKEX), die in regelmäßigen Abständen in Deutschland durchgeführt wird. \\
Der Ablauf der strategischen Krisenübung lässt sich in vier Abschnitte unterteilen:

\begin{itemize}
    \item \underline{Übungsplanung:}
    Es erfolgt eine konzeptionelle Vorbereitung. Im Übungsrahmen, dem zentralen Dokument, werden das Thema, die Beteiligten, Ziele, Organisationseinheiten und Kosten der Übung festgelegt. Es wird ein grobes Übungsszenario entwickelt und die Grundzüge der Übungsauswertung werden gesetzt.  \
    
    \item \underline{Übungsvorbereitung:}
    Es erfolgt der Aufbau der Übungssteuerung, des Kommunikationsplans und es wird das Konzept des Übungsszenarios weiterentwickelt. Dabei wird ein Drehbuch entwickelt, welches den chronologischen Verlauf der Übung festhält. Die Kernelemente des Szenarios sind dabei der allgemeine Hintergrund, die fiktive Ausgangslage, die fiktive Lageentwicklung sowie jegliche Annahmen und Besonderheiten. Des Weiteren wird ein Auswertungskonzept inklusive Auswertungsunterlagen entwickelt und die Inhalte der Vorbesprechung sowie Anweisungen für Teilnehmenden werden bestimmt.\

    \item \underline{Übungsdurchführung:}
    Vor der Durchführung ist eine Abstimmung aller Maßnahmen essenziell, um mögliche Missverständnisse zu vermeiden. Es erfolgt anschließend eine Lageeinweisung aller Beteiligten in ihre Rollen und die Überprüfung der Technik. Im Anschluss kann die geplante Übung durchgeführt und dokumentiert werden. 
    Bei frei verlaufenden Übungen sollte keine Korrektur der Entscheidungen erfolgen und die Übung sollte auch bei abweichendem Verlauf nicht unterbrochen werden, solange die fiktive Lage und das Szenario beachtet werden.
\

    \item \underline{Übungsauswertung:}
    Nach dem Durchlauf der Übung werden Erfahrungsberichte, Dokumentation, Fragebögen und Berichte von Beobachtern in einem zentralen Workshop behandelt und ein Auswertungsbericht erstellt.\\    

\end{itemize} 


Der Behelf für das „Anlegen und Durchführen von Einsatzübungen“ des \textbf{Bundesamts für Bevölkerungsschutz der Schweizerischen Eidgenossenschaft} \autocite{Behelf_Einsatzübungen} ist eine Unterlage, die zur Ausbildung dient und neben der Anleitung auch Formulare und Textvorlagen zur Verfügung stellt. \\
Der erste Teil der Anleitung behandelt das Anlegen einer Übung. Es wird der Bedarf für eine Übung ermittelt und es werden Thema, Ziele, Gelände sowie Übungsobjekte der Übung bestimmt. Anschließend wird ein Konzept erstellt. Das Konzept soll auf Ausführbarkeit überprüft und je nach Bedarf nur als Konzept oder auch mit detaillierteren Unterlagen dokumentiert werden. \\
Der zweite Teil geht auf die Durchführung einer Übung ein. Die Übungsleitung wird in klar zugewiesene Aufgabenbereiche eingeteilt, wobei die Beteiligten die nötige Fachkompetenz aufweisen sollten. Kommunikationsmittel und genutzte Markierungen/Signaturen müssen festgelegt sein. Auf das Durchführen einer Übung folgt eine Besprechung, die eine Bilanzierung, Erläuterung von Zusammenhängen, Bewertung und Würdigung der Arbeit sowie darauffolgende künftige Lehren behandeln soll. Die durchgeführte Übung wird anschließend mit der AEK-Methode (Aussage-Erkenntnis-Konsequenz) ausgewertet. \ 


Das \textbf{Deutsche Rote Kreuz} bietet im Buch „Durchführung und Auswertung von MANV-Übungen“ \autocite{MANV-Übungen} eine Bandbreite an Konzepten und Umsetzungshilfen für Übungen in Bezug auf das Szenario des Massenanfalls von Verletzten (MANV).
Auch hier beginnt der Übungsablauf mit der Planung. Ziele, Schadenslage, Einsatzmittel, Verlauf und das Budget werden festgelegt. \\
In der anschließenden Vorbereitungsphase werden die Beteiligten benannt und die Kommunikation abgestimmt. Da bei dieser Art von Übung eine hohe Zahl an „Mimen“ (Figuranten, die im Szenario verletzte Personen darstellen) nötig ist, erfolgt in diesem Schritt die Registrierung, Terminabklärung sowie die Erstellung eines Zeitplans für den Übungstag. \\

\ 

Im nächsten Schritt folgt die Durchführung der Übung. Es erfolgt eine Einweisung und Sicherheitsbelehrung für die Beteiligten. Empfohlen werden pro Übung zwei Übungsläufe. \\
Nach der Durchführung erfolgt erst eine direkte Nachbereitung, bestehend aus einer Selbsteinschätzung und einer Vorstellung der Bewertungsindikatoren und Übungsdaten. Zuletzt folgt die spätere Nachbereitung, bei der die Übungsdaten in einer Führungskräftenachbesprechung evaluiert werden und anschließend ein Abschlussbericht an alle Beteiligten versendet wird. \\

Das Begleitheft zur Unterstützung der Unfallverhütung beim Übungs- und Schulungsdienst der \textbf{Arbeitsgemeinschaft der Feuerwehr-Unfallkassen} \autocite{Feuerweh-Unfallkasse} beschreibt einen empfohlenen Ablauf einer Einsatzübung. Dieser Ablauf gleicht im Allgemeinen den bereits aufgeführten Inhalten. \\
Es erfolgt eine Übungsvorbereitung mit Gefährdungsbeurteilung, technischer und organisatorischer Planung, und Vorbesprechung. Darauf folgt die Durchführung und anschließend wird eine Nachbereitung mit Nachbesprechung und Auswertung durchgeführt. \\ 
Zusätzlich sind Hinweise bezgl. der Übungsdurchführung genannt: \\
Der Übungsort soll auf Gefahrenquellen überprüft und ausreichend beleuchtet sein, die Ausrüstung soll vor Beginn überprüft werden, Übungsteilnehmer sollen nicht überlastet werden und auf besonders gefährliche Handlungen soll verzichtet werden. Der Übung soll dabei dieselbe Aufmerksamkeit wie einem realen Einsatz geschenkt werden. \\
Häufige Fehler treten bei Einsatzübungen durch mangelnde Vorbereitung, falsche Annahmen, fehlender Berücksichtigung von vorhandenem Fachwissen und Einbeziehung nicht relevanter Faktoren auf. Außerdem werden oft Aufgabenbereiche im Vorhinein zu detailliert erläutert und Darstellungen unzureichend bzw. nicht erkennbar einbezogen. \autocite{Feuerweh-Übungspräsentation}

Ein weiterer Anwendungsbereich, in dem Sicherheitsübungen erfolgen, ist das Fahrsicherheitstraining. Der Ablauf in der Praxis einschließlich des behandelten Inhalts, ist abhängig vom Veranstalter und Fahrzeugtyp. Zum Großteil besteht diese Übung aus einem Theorie-Teil, einem Praxis-Teil und einer Nachbesprechung. \autocite{DVR}

\newpage
\subsection{Wissenschaftliche Fachliteratur}
\label{chap:scientific_research}
Gegenwärtig gibt es Forschungsprojekte,  die immersive Software für das Training bzw. die Vorbereitung auf kritische Situationen getestet haben. Es gibt beispielsweise bereits eine Lehrsoftware für Grundschullehrer, die mittels virtueller Realität das richtige Verhalten in Brandsituationen schulen soll. Die Software beinhaltet verschiedene Lernmechanismen wie dynamische Geschichten, Realismus, Bewegungsfreiheit, Level- und Punktsysteme \autocite{DDE_of_VR}. In der Fertigungsindustrie gibt es bereits Alarm-Management-Systeme, die in VR-Anwendungen integriert wurden. Das System verbessert die Handhabung von Alarmen und soll dadurch die Sicherheit von Arbeitnehmern erhöhen. Das System kann Alarme automatisch auslösen und zeigt Benutzern visuelle Warnungen an, wenn ein Alarm ausgelöst wird \autocite{Design_of_VR-training}.

Darüber hinaus bestehen wissenschaftliche Studien zu VR/AR-Anwendungen, welche untersuchen, ob und inwieweit sich die Vorbereitung auf Notfälle, die Reaktion während eines Notfalls und die Erholung nach einem Notfall durch den Einsatz von VR und AR-Anwendungen verbessert \autocite{VRandAR}. 

Einige Produktionsmaschinen und Systeme sind so vernetzt, dass sie Maschinendaten selbständig auswerten und in der Cloud als 3D-Modell visualisieren können. Die Maschine gibt also selbst Rückmeldungen über ihren Status und kann ggf. Abweichungen und Fehler autonom melden \autocite{Mascienenausf_entdecken}.

Ein weiterer Ansatz besteht darin, die Gerätewartung und Diagnose mittels Augmented Reality zu verbessern. Durch die Nutzung des Systems konnten Studierende Datenanalysen und -erfassung für Wartungsanwendungen leichter ausführen. Das Verständnis der Studierenden in diesem Sachbereich wurde durch die Einbindung von AR positiv beeinflusst (\cite{Develop_and_Asses_AR}). 

\newpage




\newpage

%\subsection{Literaturverweise}
%[1] Mystakidis, S., Besharat, J., Papantzikos, G., Christopoulos, A., Stylios, C., Agorgianitis, S. and Tselentis, D. (2022). Design, Development, and Evaluation of a Virtual Reality Serious Game for School Fire Preparedness Training. Education Sciences, 12(4), p.281. doi:https://doi.org/10.3390/educsci12040281. \\

%[2] Matsas, E. and Vosniakos, G. (2015). Design of a virtual reality training system for human–robot collaboration in manufacturing tasks. [online] Available at: https://www.researchgate.net/publication/271837242_Design_of_a_virtual_reality_training_s ystem_for_human-robot_collaboration_in_manufacturing_tasks [Accessed 18 May 2023]. \\


%[3] Zhu, Y. and Li, N. (2021). Virtual and Augmented Reality Technologies for Emergency Management in the Built Environments: A State-of-the-Art Review. Journal of Safety Science and Resilience, Volume 2(March 2021). doi:https://doi.org/10.1016/j.jnlssr.2020.11.004. \\


%[4] Weidle, D. (2023). Maschinenausfälle entdecken bevor sie auftreten. Maschinenausfälle entdecken bevor sie auftreten, 26(5), pp.26–27. doi:https://doi.org/10.1002/citp.202300515. \\

%[5] Shyr, W.-J., Tsai, C.-J., Lin, C.-M. and Liau, H.-M. (2022). Development and Assessment of Augmented Reality Technology for Using in an Equipment Maintenance and Diagnostic System. Sustainability, 14(19), p.12154. doi:https://doi.org/10.3390/su141912154. \\


%[6] Yan, W., Wang, J., Lu, S., Zhou, M. and Peng, X. (2023). A Review of Real-Time Fault Diagnosis Methods for Industrial Smart Manufacturing. Processes, 11(2), pp.369–369. doi:https://doi.org/10.3390/pr11020369.
 \clearpage
\section{Vorgehen}

In dem folgenden Abschnitt wird das Vorgehen der Projektgruppe zur Realisierung des Alarmszenarios erläutert. Hierzu wird auf den Arbeitsmodus  und erarbeitete Arbeitsergebnisse eingegangen.

Die Zielsetzung der Projektarbeit \textit{Alaarm-ZIP} lag in der Schaffung eines interaktiven Alarmsimulation, in welcher Übungsteilnehmer fiktive Alarmszenarien im Kontext einer vernetzten Industrie 4.0 Anlage durchlaufen. Die Szenarien werden durch Effekte wie dem Leuchten einer LED-Lampe realisiert, welche beim Übungsteilnehmer einen immersiven Effekt hervorrufen sollen. 

Um die Zielsetzung des Projektes zu realisieren, wurde zum Start des Projektes in der siebenköpfigen Projektgruppe zunächst Themen der Projektorganisation, etwa der Arbeitsmodus und die eingesetzen Tools zum Projektmanagement, festgelegt. Es wurde ein wöchentlicher Sync-Termin mit der Länge von 1 Stunde etabliert. Dieser Termin diente zur gemeinsamen Diskussion der Ergebnisse der vorausgegangenen Woche, der Definition und Festlegung neuer Arbeitspakete für die kommende Woche und der Abstimmung von Terminen in Präsenz zur Erarbeitung größerer Aufgaben von besonderer Bedeutung für den Projektfortschritt. An den Projektstart schloss in der darauffolgenden Woche ein Präsenztermin an, um mithilfe von erarbeiteten Rechercheergebnissen zu verwandten Themen von Alarmszenarien und -simulationen die eigene Aufgabenstellung mittels Brainstorming zu vertiefen. Die Ergebnisse wurden als Mindmap mit dem Kollaborationstool Miro festgehalten. Daran anschließend wurden konkrete Szenarioideen gebildet, welche Gedanken zu Ablauf, Immersionsgeräten und der Administration des Alarmszenarios beinhalten. 

Auf Grundlage der bisher erarbeiteten Ergebnisse wurde sich als Team auf ein Szenario geeinigt, welches als \textit{Minimal Viable Product} (\textit{MVP}) für die Projektarbeit zu realisieren ist. In einer weiteren Arbeitssitzung in Präsenz wurden nun ein Ablaufdiagramm des MVPs in der Prozessmodellierungssprache \textit{BPMN} entwickelt. Unter Abstimmung des aufgestellten Modells mit der Projektleitung von Seiten des Lehrstuhls wurden die technischen Rahmenbedingungen diskutiert. Durch diese Sitzung wurden schlussendlich die technische Implementierung realisiert, um das MVP-Szenario in der Simulationsumgebung des Zentrums Industrie 4.0 am Lehrstuhl für Wirtschaftsinformatik, Prozesse und Systeme der Universität Potsdam zu integrieren. Über den Projektzeitraum hinweg wurde kontinuierlich die hiermit vorliegende Projektdokumentation aktualisiert und weitere Dokumente, wie Beschreibungskarten und Sicherheitsdokumente zur Verwendung für die Alarmsimulation geschaffen.   



\begin{comment}
\section{Beschreibungskarte}


\ 

\textbf{AGENDA} 

\ 

\textbf{3.1 Einleitung und Szenario-Überblick} 
\begin{itemize}
\item
Präsentation des Szenarios und seiner Komponenten
\item
Erklärung des Ablaufs und der Eskalationsstufen
\end{itemize}

\

\textbf{3.2 Sicherheitsbriefing} 
\begin{itemize}
\item
Durchsicht der Gesundheits- und Sicherheitshinweise
\end{itemize}

\

\textbf{3.3 Bedienungsanleitung und Aktivierung der Eskalationsstufe}
\begin{itemize}
\item
Erläuterung der Bedienung der Maschine und des Tablets
\item
Aktivierung der Eskalationsstufe 
\end{itemize}

\

\textbf{3.4 Anzeige der Fehlermeldung und Lösungsansatz }
\begin{itemize}
\item
Erläuterung der Schritte zur Lösung der Fehlermeldungen
\end{itemize}

\

\textbf{3.5 Eskalationsstufe 1}
\begin{itemize}
\item
Durchgehen der Eskalationsstufe 1 und der damit verbundenen Entstörungsvorgang
\item 
Einschreiten der Zufallsvariable
\item
Diskussion über die möglichen Auswirkungen der Zufallsvariable auf den Ausgang des Szenarios
\end{itemize}

\

\textbf{3.6 Eskalationsstufe 2}
\begin{itemize}
\item
Durchgehen der Eskalationsstufe 2 und der damit verbundenen Entstörungsvorgang
\item
Einschreiten der Zufallsvariable
\item
Diskussion über die möglichen Auswirkungen der Zufallsvariable auf den Ausgang des Szenarios
\end{itemize}

 \

\textbf{3.7 Eskalationsstufe 3}
\begin{itemize}
\item
Durchgehen der Eskalationsstufe 2 und der damit verbundenen Entstörungsvorgang
\item
Einschreiten der Zufallsvariable
\item
Diskussion über die möglichen Auswirkungen der Zufallsvariable auf den Ausgang des Szenarios
\end{itemize}

\

\textbf{3.8 Voraussichtliche Zeitangaben des Szenarios zum Abschluss}
\begin{itemize}
\item
Messung der Durchlaufzeit bei optimalen Abläufen
\end{itemize}

\newpage

\subsection{Einleitung und Szenario-Überblick}

In diesem Szenario wird ein komplexes System aus verschiedenen Komponenten und Prozessen beschrieben, das in einen Maschinen-Simulationsraum implementiert wird. Die Szenariokomponenten bestehen aus den Teilnehmern, Tablet, Subwoofer/Lautsprecher, Wärmelampe/Licht und Nebelmaschine. Diese Komponenten sind in einer Umgebung platziert, in der es Anlagen zum Bedienen gibt, die alle durch ein Förderband miteinander verbunden sind. Das System ist so konzipiert, dass es drei Eskalationsstufen gibt, die nacheinander aktiviert werden können. Jede Eskalationsstufe beinhaltet einen Entstörungsvorgang, der vom Teilnehmer gelöst werden muss. Dabei werden verschiedene Szenariokomponenten aktiviert, um den Prozess zu unterstützen und zu simulieren. Teilnehmer des Szenarios, die bei einem Enstörungsvorgang versagen, werden automatisch zur nächsten Eskalationsstufe befördert, bis sie maximal die Eskalationsstufe 3 erreichen. Während eines Enstörungsvorgangs steht den Teilnehmern dabei nur ein Versuch zur Verfügung. Allerdings haben erfolgreiche Szenarioteilnehmer die Möglichkeit, nach Abschluss einer beliebigen Eskalationsstufe das gesamte Szenario vorzeitig zu beenden. Pro Durchgang führt nur ein Szenarioteilnehmer das Szenario durch. Es gibt einen Betreuer, der diese Vorgänge beobachtet und bei Notfällen einschreitet, jedoch ist er nicht für die gesamten Handlungen im Szenarioprozess eingeplant.

\

\underline{Einleitung zum Szenario:} \\
Der Teilnehmer aktiviert die Anlage mit einem Startknopf auf dem Tablet.\hspace{0pt}\marginpar{\footnotesize{ca. 1 Sek.}}
Es wird das Datenpaket für den Szenarioteilnehmer ausgelesen, woraufhin Anweisungen wie Gesundheitswarnungen oder weitere Verfahren angezeigt werden, die der Teilnehmer lesen soll:
\



\

\underline{Einleitung der Eskalationsstufe 1:}

Die Anlage läuft an. \hspace{0pt}\marginpar{\footnotesize{ca. 8 Sek.}}
Gleichzeitig werden in der ersten Stufe die Szenariokomponenten aktiviert, wie der Subwoofer/Lautsprecher, der einen langsamen pulsierenden Ton erzeugt,\hspace{0pt}\marginpar{\footnotesize{ca. 1 Sek.}} sowie die Wärmelampe/Licht (Andon-Signalleuchte), die an der jeweiligen Maschine aktiviert wird und kontinuierlich leuchtet.
Die Anlage ruckelt und stoppt ihren Arbeitsfluss. \hspace{0pt}\marginpar{\footnotesize{ca. 5 Sek.}}

\

\subsection{Anzeige der Fehlermeldung und Lösungsansatz}

\hspace{0pt}\marginpar{\footnotesize{ca. 40 Sek.}} Auf dem Tablet erscheint der Name der betroffenen Maschine, zu der der Szenarioteilnehmer hinlaufen muss:

\

\emph{FEHLERCODE: Machine01}\\
\\

Hinlaufen zu dieser Maschine. \hspace{0pt}\marginpar{\footnotesize{ca. 15 Sek.}}



\



\\
\hspace{0pt}\marginpar{\footnotesize{ca. 45 Sek.}}Auf dem Tablet erscheint ein QR-Code, der gescannt werden muss, um einen Entstörungsvorgang in 40 Sekunden bzw. das Problem zu lösen wie:\

\

Ein großer Button in der Mitte mit der Aufschrift: "Mit Maschine interagieren". \
Der Button verschwindet und macht Platz für den QR-Code-Leser, der wie ein rechteckiges Fenster in der Mitte des Bildschirms erscheint. Wenn der gescannte Code dem erwarteten Code entspricht: Eine Nachricht wird angezeigt, die besagt: "Dies ist der richtige Code! Starte Spiel...".
Wenn der gescannte Code nicht dem erwarteten Code entspricht: Eine rote Fehlermeldung wird angezeigt, die besagt: "Falscher Code! Gehen Sie zu Machine03." Nach einer kurzen Verzögerung von 2 Sekunden wird der QR-Code-Leser erneut aktiviert, um einen weiteren Scan zu ermöglichen.
\



\\


\subsection{Eskalationsstufe 1}


\

\underline{\emph{Numbers-basic}} \\
\emph{Auf dem Tablet wird eine Notiz angezeigt:} \\

\

\emph{(!) Die Werkschritt-Reihenfolge der Maschine scheinen durcheinander gebracht zu sein. } \\

\emph{Klicke auf die Nummern 1 bis 10 in aufsteigender Reihenfolge, um die richtige Reihenfolge wiederherzustellen. Für die Operation sind nur 10 Sekunden vorgesehen!} \\

\emph{3, 5, 8, 10, 9, 2, 7, 4, 1, 6} \\

\


Es folgt eine Verzweigung, die eine Zufallsvariable beinhaltet:

{Nach erfolgreichem Abschluss des Entstörungsvorgangs wird auf dem Tablet eine Benachrichtigung angezeigt, die den Abschluss des Vorgangs bestätigt. Um die Wahrscheinlichkeit von 33\% zu simulieren, wird eine digitale Würfelfunktion verwendet, die die Zahlen 1, 2 und 3 generiert und damit drei mögliche Zustände repräsentiert. Nach dem Entstörungsvorgang wird die digitale Würfelfunktion ausgeführt. Wenn die gewürfelte Zahl eine 1 oder 2 ist (entspricht einer Wahrscheinlichkeit von 2/3), wird auf dem Tablet angezeigt, dass das Problem als gelöst erfasst wurde. Falls die gewürfelte Zahl eine 3 ist (entspricht einer Wahrscheinlichkeit von 1/3), wird auf dem Tablet angezeigt, dass das Problem nicht vollständig gelöst wurde. \\

\begin{itemize}
\item
Sollte der Entstörungsvorgang innerhalb der vorgegebenen Zeit nicht erfolgreich abgeschlossen werden oder eine falsche Eingabe getätigt werden, erfolgt der Übergang zur zweiten Eskalationsstufe automatisch.
\item
Ist der Entstörungsvorgang erfolgreich gelöst, folgt die Zufallsvariable, dabei wird bei einer 1/3 Wahrscheinlichkeit das Problem als nicht gelöst erfasst. Dann erfolgt ebenfalls der Übergang zur zweiten Eskalationsstufe automatisch.
\item
\hspace{0pt}\marginpar{\footnotesize{ca. 3 Sek.}}Wenn der Entstörungsvorgang erfolgreich gelöst ist, folgt die Zufallsvariable, bei der mit einer Wahrscheinlichkeit von 2/3 das Problem als gelöst erfasst wird. Wenn dieses Szenario eintrifft, werden alle Szenariokomponenten ausgeschaltet und das Szenario für den Teilnehmer ist erfolgreich beendet.
\end{itemize}


\newpage 

\subsection{Eskalationsstufe 2}

Simultan werden die Szenariokompontenten Subwoofer/Lautprecher mit einem mittelschnell schlagenden Ton aktiviert. Der Vaporizer stößt derweil Brandgeruch aus. Die LED-Leuchte flackert bei der Maschine blau. Diese bleiben durchgehend in der zweiten Eskalationsstufe aktiv.\hspace{0pt}\marginpar{\footnotesize{ca. 5 Sek.}} \\
Der Szenarioteilnehmer soll sich selbstständig zur nächsten Maschine bewegen, was durch eine Andon-Signalleuchte an der Maschine signalisiert wird. (Zeit: ca. 15 Sekunden) \\
\hspace{0pt}\marginpar{\footnotesize{ca. 40 Sek.}}Erneut muss ein Entstörungsvorgang durchgeführt werden: 

\

\underline{\emph{Reaction}} \\
\emph{(!) Auf dem Tablet wird eine Notiz angezeigt:}\\

Problem mit der Kalibrierung! Etwas scheint mit der Kalibrierung der Maschine ein Problem zu geben, das schwerwiegende Fehler auslöst. Um alle möglichen Fehlerquellen auszuschließen, müssen die Eingaben neu kalibriert werden. \\
          
Zur korrekten Kalibrierung muss der richtige grün aufleuchtende Bereich innerhalb von 0,6 Sekunden gedrückt werden. Die Bereiche fangen in einem schwarzen Zustand an. Falls du länger als 0,6 Sekunden brauchst, um den grün aufleuchtenden Bereich zu drücken, kann die Maschine nicht kalibriert werden. Es benötigt 6 erfolgreiche Kalibrierungen.

\
\includegraphics[width=1.6cm, height=1.3cm]{res/2cube.jpg} \\
\includegraphics[width=3cm, height=2cm,]{res/cube.jpg} 


\

Danach erfolgt eine weitere Verzweigung, bei der die Zufallsvariable aktiviert wird. \\

\

\subsection{Eskalationsstufe 3} \\

Gleichzeitig werden die Szenariokomponenten Subwoofer/Lautsprecher mit einem schnell schlagenden Ton, der Vaporizer mit einem intensiveren Geruch als zuvor und die Wärmelampe aktiviert. Die Andon-leuchte flackert rot. Diese bleiben durchgehend in der dritten Eskalationsstufe aktiv.\hspace{0pt}\marginpar{\footnotesize{ca. 5 Sek.}} \\
Der Szenarioteilnehmer soll zur nächsten Maschine, welches durch eine Andon-Signalleuchte signalisiert wird, hinbewegen.\hspace{0pt}\marginpar{\footnotesize{ca. 15 Sek.}} \\
Erneut muss ein Entstörungsvorgang durchgeführt werden: \hspace{0pt}\marginpar{\footnotesize{ca. 40 Sek.}}

\

\underline{\emph{Numbers-advanced}} \\
\emph{Auf dem Tablet wird eine Notiz angezeigt:} \\

\

\emph{(!) Die Werkschritt-Reihenfolge der Maschine scheinen durcheinander gebracht zu sein. } \\

\emph{Der Algorithmus der Maschine scheint ein Problem zu haben. Mehrere Prozesse scheinen separat zu laufen, die wieder in die richtige Reihenfolge gebracht werden müssen.
        Klicke die Nummern 1 bis 10 in aufsteigender Reihenfolge an, um dies zu tun.
        Nach jeweils 5 und 10 Sekunden werden die Zahlen gemischt.
        Für diese Reparatur sind lediglich 15 Sekunden vorgesehen!} \\


\emph{3\hspace{15pt} 5\hspace{42pt} 8\hspace{58pt} 10\hspace{75pt} \\9\hspace{25pt} 2\hspace{37pt} 7\hspace{83pt} \\4\hspace{39pt} 1\hspace{66pt} 6\hspace{81pt}} \\

\
Danach erfolgt eine weitere Verzweigung, bei der die Zufallsvariable aktiviert wird.

\

\subsection{Voraussichtliche Zeitangaben des Szenarios}
Sämtliche Zeitangangaben beruhen auf einer Schätzung bzw. theoretischen Planung und sind je nach individueller Durchführung und Reaktionsgeschwindigkeit des Teilnehmers variabel. Diese Zeitangaben stehen daher unter Vorbehalt und dienen lediglich als grobe Orientierung. Es ist möglich, dass die tatsächliche Dauer für jeden Teilnehmer unterschiedlich ist. \\

\

\underline{Benötigte Zeit für Eskalationsstufe 1} \\
Lesen der Anweisungen: ca. 2 Minuten \\
Aktivierung der Anlage: ca. 8 Sekunden \\
Unterbrechung der Anlage ca. 5 Sekunden \\
Anzeige der Fehlermeldung: ca. 40 Sekunden \\
Hinlaufen zur Maschine: ca. 15 Sekunden \\
Entstörungsvorgang in der ersten Stufe: ca. 45 Sekunden \\
Einschreiten der Zufallsvariable: ca. 3 Sekunden \\
Geschätzte Gesamtzeit für den Abschluss der ersten Stufe: ca. 4 Minuten \\

\

\underline{Benötigte Zeit für Eskalationsstufe 2} \\
Zeit für die erste Stufe: ca. 4 Minuten \\
Hinlaufen zur Maschine: ca. 15 Sekunden \\
Entstörungsvorgang in der zweiten Stufe: ca. 40 Sekunden \\
Einschreiten der Zufallsvariable: ca. 3 Sekunden \\
Geschätzte Gesamtzeit für den Abschluss der ersten und der zweiten Stufe: ca. 5 Minuten \\

\

\underline{Benötigte Zeit für Eskalationsstufe 3 / Max. Zeit bei Erfolglosigkeit} \\
Zeit für die zweite Stufe: ca. 5 Minuten \\
Hinlaufen zur Maschine: ca. 15 Sekunden \\
Entstörungsvorgang in der dritten Stufe: ca. 40 Sekunden \\
Einschreiten der Zufallsvariable: ca. 3 Sekunden \\
Geschätzte Gesamtzeit für den Abschluss der ersten, zweiten und der dritten Stufe: ca. 6 Minuten \\
\end{comment}



 \clearpage

\includegraphics[pages=1, scale=0.4]{res/Finales Szenario BPMN(korrigiert).drawio-5.pdf}
%\includegraphics[pages=1, scale=0.45]{res/Finales Szenario BPMN(korrigiert).drawio-5.pdf}

\newpage
\section{Konzeptionierung des finalen Szenarios}

Die Konzeptionierung unseres finalen Szenarios folgte einem stringenten Vorgehensansatz, der aus mehreren aufeinanderfolgenden Schritten bestand. Im initialen Schritt führten wir ein individuelles Brainstorming über die verschiedenen Komponenten und Abläufe der Szenarien durch. Hierbei bedienten wir uns digitaler \glqq Sticky Notes\grqq{} auf einem Miro Board. In diesem kreativen, ungebundenen und explorativen Prozess wurden diverse Aspekte in Betracht gezogen. Exemplarisch verdeutlichen einige Beispiele der \glqq Sticky Notes\grqq{} die Breite der Diskussionen und Ideenfindung. Unter anderem wurde die Idee einer Alarmsignalgabe in Form von Bass oder Sirene erörtert, die nur bei der jeweiligen Maschine ausgelöst werden konnte. Dieses Signal sollte in einer Art gestaltet sein, dass es sich pulsierend wiederholt, solange die Maschine nicht repariert wurde. Abhängig von der benötigten Stressstufe sollte sich sowohl die Tonhöhe als auch die Lautstärke des Geräusches verändern. Ein weiterer Vorschlag bestand darin, Brandgeruch als Warnsignal einzusetzen. Auch visuelle Effekte wie ein schnelles, abwechselndes Licht wurden als Möglichkeit in Betracht gezogen. Zusätzlich wurde die Simulation eines Stromausfalls durch zeitweises Abschalten von Licht und Displays diskutiert. Dies würde von einem FI-Schutzschalter ausgelöst.

Im darauffolgenden Schritt erfolgte die Kategorisierung der \glqq Sticky Notes\grqq{} in verschiedene Szenarien, Komponenten und nachfolgende Aktivitäten. Die Szenarien wurden in die Kategorien \glqq Riechen\grqq{}, \glqq Hören\grqq{}, \glqq Sehen\grqq{} und \glqq Fühlen\grqq{} unterteilt. Die Komponenten wurden wiederum in solche eingeteilt, die für den Einsatz im Szenario und solche, die zur Bewältigung der Situation notwendig waren. Um die Übersichtlichkeit und die Kommunikation zu verbessern, wurde dieser Schritt persönlich durchgeführt.

Anschließend überführten wir die physisch erstellte Übersicht in digitale Form. In einer Plenumsdiskussion bewerteten alle Mitglieder die präferierten \glqq Sticky Notes\grqq{}, indem sie Punkte vergaben. Auf diese Weise konnten wir eine gemeinsame Entscheidungsgrundlage schaffen und die wichtigsten Ideen identifizieren.

Zu guter Letzt selektierten wir drei Szenarien basierend auf den \glqq Sticky Notes\grqq{} mit den höchsten Punktzahlen. Das erste Szenario umfasste den Einsatz von visuellen Effekten mit flackerndem Licht, einem an das Stresslevel angepassten Geräusch, LED-Streifen und Lautsprechern. Im zweiten Szenario lag der Schwerpunkt auf dem Einsatz von Brandgeruch, begleitet von visuellen Effekten wie Scheinwerfern in rotem Licht, einer Infrarotlampe und dem Erzeugen von Rauch mithilfe einer Nebelmaschine. Auch hier wurden LED-Streifen verwendet. Das dritte Szenario beinhaltete Schläge oder Stöße, die den Bruch einer Maschine simulieren sollten. Das begleitende Geräusch sollte ebenfalls je nach Stresslevel variieren und wurde durch einen Subwoofer und Lautsprecher unterstützt.

Dieser systematische Ansatz bei der Erstellung des finalen Szenarios ermöglichte es uns, die verschiedenen Ideen zu erfassen, zu strukturieren und zu bewerten.

Aus dem Entwurf der drei Alarmszenarien haben sich abschließend diverse Vorgehensweisen und spezifische Komponenten manifestiert, die für die Realisierung des Projekts unabdingbar sind. Inbegriffen sind hierbei die Interaktionsplattformen sowie die damit einhergehende Technik, Entstörungsverfahren, Kommunikationsschemata und audiovisuelle Signaltechniken.

Die nun spezifizierten Szenarioansätze und die damit verknüpften Bestandteile wurden als BPMN-Modelle entworfen und Herrn Panzer zur Begutachtung vorgestellt. Die individuellen Szenariomodelle repräsentieren schematisch den Ablauf der Szenarien und ermöglichen eine mühelose Verfolgung der divergenten möglichen Verläufe, Resultate sowie ihrer Kombinationen und entsprechenden Auslöser.

Im Anschluss wurden die modellierten Szenarien in ein zentralisiertes Modell transferiert, um die besten Merkmale und Ansätze aller Szenarien zu kombinieren. Hierzu hat jedes Mitglied der Gruppe in den drei zugrunde liegenden Szenariomodellen für individuelle Komponenten Punktzahlen vergeben und gekennzeichnet. Diese Einschätzungen wurden einer eingehenden Auswertung unterzogen, wobei die Bestandteile mit den höchsten Punktzahlen in das Hauptmodell integriert wurden.

Die Gruppe hat sich für das Projekt auf wöchentliche Meetings verständigt, in denen Ergebnisse abgeglichen und korrigiert werden sollten, sowie die Zuweisung von einzelnen Tasks erfolgte. Die Meetings fanden jeweils mittwochs gegen 17:00 Uhr deutscher Zeit statt.
\\
\label{section:name-} \clearpage
\section{Technische Dokumentation}

Im folgendem Abschnitt wird auf die Implementierung des Alarmszenarios und dazugehörige Überlegungen eingegangen. Zu Beginn wird der gewählte Teckstack dargelegt und durch Ausführungen zur Systemarchitektur ergänzt. 
Daran schließt ein Abschnitt zur Erklärungen der Konfigurationen von Alarmszenarien als auch eingesetzten APIs an.

\subsection{Techstack}

Die Auswahl des Teckstacks zur Implementierung von ALAARM-ZIP hängt zentral von einer Vielzahl von Vorbedingungen ab.

Bezüglich der verwendeten Programmiersprache wurde sich für den Einsatz von TypeScript entschieden. Neben der Eignung für Webentwicklung

Durch die Vorbedingungen des Einsatzes von Tablets durch den Administrator zur Konfiguration und Steuerung, der ähnlichen Anwendung durch Simulationsteilnehmer zur Lösung von Rätseln und der Bereitstellung von QR-Codes auf Bildschirmen der simulierten Maschinen

In Anbetracht der Tatsache, dass einerseits Administratoren mithilfe von Tablets Szenarien konfigurieren und steuern können und Teilnehmer der Übungssimulation

\subsection{Systemarchitektur}

\subsection{Konfiguration von Alarmszenarien}

\subsection{APIs} \clearpage
%\input{/pfad/zum/inhalt} \newpage
\setlength\bibitemsep{12pt}
\printbibliography[title=Literaturverzeichnis]

%\section*{Anhang}
\label{sec:anhang}

\subsection*{Sicherheitsbriefing}\hspace{0pt}\marginpar{\footnotesize{ca. 2 Min.}}
\emph{Bitte nehmen Sie sich die Zeit, diese Anweisungen sorgfältig zu lesen und zu verstehen, bevor Sie mit dem Betrieb der Maschine fortfahren. Ihre Sicherheit hat oberste Priorität.}

\emph{WICHTIG: GESUNDHEITSWARNUNG} \\

\emph{Bitte stellen Sie sicher, dass Sie die geeignete Schutzkleidung tragen, bevor Sie mit dem Betrieb dieser Maschine beginnen. Dies beinhaltet Schutzhandschuhe, Schutzbrille und nach Anweisung des Übungsleiters eine Atemschutzmaske. 
Es besteht die Möglichkeit, dass während des Szenarios Rauch oder andere intensiv riechende Substanzen verwendet werden.
Dieses Szenario beinhaltet flackernde Lichter, die bei einigen Personen mit Epilepsie zu Anfällen führen können. Bitte beachten Sie, dass Ihre Gesundheit und Sicherheit oberste Priorität haben. Falls Sie Bedenken haben oder gesundheitliche Einschränkungen haben, die sich auf Ihre Teilnahme an diesem Szenario auswirken könnten, empfehlen wir Ihnen, vorab Rücksprache mit einem Arzt zu halten, um sicherzustellen, dass Sie angemessen geschützt sind und das Szenario ohne Risiken für Ihre Gesundheit durchführen können.}

\subsection*{Beschreibungskarte MVP}

In diesem Szenario wird ein Ablauf aus verschiedenen Komponenten und Eskalationsstufen beschrieben, das in der Simulationsumgebung des Zentrums Industrie 4.0 implementiert wurde. Die Szenariokomponenten bestehen aus den Teilnehmern, Tablet, Subwoofer/Lautsprecher, Wärmelampe/Licht und Nebelmaschine. Diese Komponenten sind in einer Umgebung platziert, in der es Anlagen zum Bedienen gibt, die alle durch ein Förderband miteinander verbunden sind. Das System ist so konzipiert, dass es drei Eskalationsstufen gibt, die nacheinander aktiviert werden können. Jede Eskalationsstufe beinhaltet einen Entstörungsvorgang, der vom Teilnehmer gelöst werden muss. Dabei werden verschiedene Szenariokomponenten aktiviert, um den Prozess zu unterstützen und zu simulieren. Teilnehmer des Szenarios, die bei einem Enstörungsvorgang versagen, werden automatisch zur nächsten Eskalationsstufe befördert, bis sie maximal die Eskalationsstufe 3 erreichen. Während eines Enstörungsvorgangs steht den Teilnehmern dabei nur ein Versuch zur Verfügung. Allerdings haben erfolgreiche Szenarioteilnehmer die Möglichkeit, nach Abschluss einer beliebigen Eskalationsstufe das gesamte Szenario vorzeitig zu beenden. Pro Durchgang führt nur ein Szenarioteilnehmer das Szenario durch. Es gibt einen Übungsleiter, der diese Vorgänge beobachtet und bei Notfällen einschreitet, jedoch ist er nicht für die gesamten Handlungen im Szenarioprozess eingeplant.

\

\underline{Einleitung zum Szenario:} \\
Der Teilnehmer aktiviert die Anlage mit einem Startknopf auf dem Tablet.\hspace{0pt}\marginpar{\footnotesize{ca. 10 Sek.}}

\

\

\underline{Einleitung der Eskalationsstufe 1:}

Die Anlage läuft an. \hspace{0pt}\marginpar{\footnotesize{ca. 8 Sek.}}
Gleichzeitig werden in der ersten Stufe die Szenariokomponenten aktiviert, wie der Subwoofer/Lautsprecher, der einen langsamen pulsierenden Ton erzeugt,\hspace{0pt}\marginpar{\footnotesize{ca. 1 Sek.}} sowie die Wärmelampe/Licht (Andon-Signalleuchte), die an der jeweiligen Maschine aktiviert wird und kontinuierlich leuchtet.
Die Anlage ruckelt und stoppt ihren Arbeitsfluss. \hspace{0pt}\marginpar{\footnotesize{ca. 5 Sek.}}

\

\subsection*{Anzeige der Fehlermeldung und Lösungsansatz}

\hspace{0pt}\marginpar{\footnotesize{ca. 40 Sek.}} Auf dem Tablet erscheint der Name der betroffenen Maschine, zu der der Szenarioteilnehmer hinlaufen muss:

\

\emph{FEHLERCODE: E5432}\\
\\

Hinlaufen zu dieser Maschine. \hspace{0pt}\marginpar{\footnotesize{ca. 15 Sek.}}



\




\hspace{0pt}\marginpar{\footnotesize{ca. 45 Sek.}}Auf dem Tablet erscheint ein QR-Code Scanner, der zum Scannen des QR Codes auf der Maschine benutzt werden muss, um einen Entstörungsvorgang in 40 Sekunden bzw. das Problem zu lösen. Eine Nachricht wird angezeigt, die besagt: "Dies ist der richtige Code! Starte Spiel...".
Wenn der gescannte Code nicht dem erwarteten Code entspricht: Eine rote Fehlermeldung wird angezeigt, die besagt: "Falscher Code! Gehen Sie zu Machine03." Nach einer kurzen Verzögerung von 2 Sekunden wird der QR-Code-Leser erneut aktiviert, um einen weiteren Scan zu ermöglichen.
\


\subsection*{Eskalationsstufe 1}

\

\emph{Auf dem Tablet wird eine Notiz angezeigt:} \\
\emph{(!) Die Werkschritt-Reihenfolge der Maschine scheinen durcheinander gebracht zu sein. } \\

\emph{Klicke auf die Nummern 1 bis 10 in aufsteigender Reihenfolge, um die richtige Reihenfolge wiederherzustellen. Für die Operation sind nur 10 Sekunden vorgesehen!} \\

\emph{3, 5, 8, 10, 9, 2, 7, 4, 1, 6} \\

\
Es folgt eine Verzweigung, die eine Zufallsvariable beinhaltet:

{Nach erfolgreichem Abschluss des Entstörungsvorgangs wird auf dem Tablet eine Benachrichtigung angezeigt, die den Abschluss des Vorgangs bestätigt. Um die Wahrscheinlichkeit von 33\% zu simulieren, wird eine digitale Würfelfunktion verwendet, die die Zahlen 1, 2 und 3 generiert und damit drei mögliche Zustände repräsentiert. Nach dem Entstörungsvorgang wird die digitale Würfelfunktion ausgeführt. Wenn die gewürfelte Zahl eine 1 oder 2 ist (entspricht einer Wahrscheinlichkeit von 2/3), wird auf dem Tablet angezeigt, dass das Problem als gelöst erfasst wurde. Falls die gewürfelte Zahl eine 3 ist (entspricht einer Wahrscheinlichkeit von 1/3), wird auf dem Tablet angezeigt, dass das Problem nicht vollständig gelöst wurde. \\

\begin{itemize}
\item
Sollte der Entstörungsvorgang innerhalb der vorgegebenen Zeit nicht erfolgreich abgeschlossen werden oder eine falsche Eingabe getätigt werden, erfolgt der Übergang zur zweiten Eskalationsstufe automatisch.
\item
Ist der Entstörungsvorgang erfolgreich gelöst, folgt die Zufallsvariable, dabei wird bei einer 1/3 Wahrscheinlichkeit das Problem als nicht gelöst erfasst. Dann erfolgt ebenfalls der Übergang zur zweiten Eskalationsstufe automatisch.
\item
\hspace{0pt}\marginpar{\footnotesize{ca. 3 Sek.}}Wenn der Entstörungsvorgang erfolgreich gelöst ist, folgt die Zufallsvariable, bei der mit einer Wahrscheinlichkeit von 2/3 das Problem als gelöst erfasst wird. Wenn dieses Szenario eintrifft, werden alle Szenariokomponenten ausgeschaltet und das Szenario für den Teilnehmer ist erfolgreich beendet.
\end{itemize}

\newpage 

\subsection*{Eskalationsstufe 2}

Simultan werden die Szenariokompontenten Subwoofer/Lautprecher mit einem mittelschnell schlagenden Ton aktiviert. Der Vaporizer stößt derweil Brandgeruch aus. Die LED-Leuchte flackert bei der Maschine blau. Diese bleiben durchgehend in der zweiten Eskalationsstufe aktiv.\hspace{0pt}\marginpar{\footnotesize{ca. 5 Sek.}} \\
Der Szenarioteilnehmer soll sich selbstständig zur nächsten Maschine bewegen, was durch eine Andon-Signalleuchte an der Maschine signalisiert wird. (Zeit: ca. 15 Sekunden) \\
\hspace{0pt}\marginpar{\footnotesize{ca. 40 Sek.}}Erneut muss ein Entstörungsvorgang durchgeführt werden: 

\emph{(!) Auf dem Tablet wird eine Notiz angezeigt:}\\

Der Prozessfluss der Maschine scheint ein Problem zu haben und zu überhitzen. Mehrere Prozesse scheinen separat zu laufen, die wieder in die richtige Reihenfolge gebracht werden müssen.
        Klicke die Nummern 1 bis 10 in aufsteigender Reihenfolge an, um dies zu tun.
        Nach jeweils 5 und 10 Sekunden werden die Prozessschritte durch eine Neukalibrierung gemischt.
        Für diese Reparatur sind aufgrund der engen Taktung lediglich 15 Sekunden vorgesehen. \\


\emph{3\hspace{15pt} 5\hspace{42pt} 8\hspace{58pt} 10\hspace{75pt} \\9\hspace{25pt} 2\hspace{37pt} 7\hspace{83pt} \\4\hspace{39pt} 1\hspace{66pt} 6\hspace{81pt}} \\

Danach erfolgt eine weitere Verzweigung, bei der die Zufallsvariable aktiviert wird. \\

\subsection*{Eskalationsstufe 3} \

Gleichzeitig werden die Szenariokomponenten Subwoofer/Lautsprecher mit einem schnell schlagenden Ton, der Vaporizer mit einem intensiveren Geruch als zuvor und die Wärmelampe aktiviert. Die Andon-leuchte flackert rot. Diese bleiben durchgehend in der dritten Eskalationsstufe aktiv.\hspace{0pt}\marginpar{\footnotesize{ca. 5 Sek.}} \\
Der Szenarioteilnehmer soll zur nächsten Maschine, welches durch eine Andon-Signalleuchte signalisiert wird, hinbewegen.\hspace{0pt}\marginpar{\footnotesize{ca. 15 Sek.}} \\
Erneut muss ein Entstörungsvorgang durchgeführt werden: \hspace{0pt}\marginpar{\footnotesize{ca. 40 Sek.}}

\emph{Auf dem Tablet wird eine Notiz angezeigt:} \\

\emph{(!) Die Werkschritt-Reihenfolge der Maschine scheinen durcheinander gebracht zu sein. } \\

\emph{Etwas scheint mit der Kalibrierung der Maschine ein Problem zu geben, wodurch schwerwiegende Fehler auslöst wurden.
                        Durch eine Fehlkalibrierung ist es zu einem Kabelbrand gekommen. Um den Schaden zu begrenzen, müssen die Eingaben schnellstmöglich neu kalibriert werden, um einen Brand zu verhindern.
                        Zur korrekten Kalibrierung muss der richtige grün aufleuchtende Bereich innerhalb von 0,6 Sekunden gedrückt werden. 
                        Die Bereiche fangen in einem schwarzen Zustand an. Falls du länger als 0,6 Sekunden brauchst, 
                        um den grün aufleuchtenden Bereich zu drücken, kann die Maschine nicht richtig kalibriert werden.
                        Es werden 6 erfolgreiche Kalibrierungen benötigt.} \\

\
Danach erfolgt eine weitere Verzweigung, bei der die Zufallsvariable aktiviert wird.

\subsection*{Voraussichtliche Zeitangaben des Szenarios}
Sämtliche Zeitangangaben beruhen auf einer Schätzung bzw. theoretischen Planung und sind je nach individueller Durchführung und Reaktionsgeschwindigkeit des Teilnehmers variabel. Diese Zeitangaben stehen daher unter Vorbehalt und dienen lediglich als grobe Orientierung. Es ist möglich, dass die tatsächliche Dauer für jeden Teilnehmer unterschiedlich ist. \\

\

\underline{Benötigte Zeit für Eskalationsstufe 1} \\
Lesen der Anweisungen: ca. 2 Minuten \\
Aktivierung der Anlage: ca. 8 Sekunden \\
Unterbrechung der Anlage ca. 5 Sekunden \\
Anzeige der Fehlermeldung: ca. 40 Sekunden \\
Hinlaufen zur Maschine: ca. 15 Sekunden \\
Entstörungsvorgang in der ersten Stufe: ca. 45 Sekunden \\
Einschreiten der Zufallsvariable: ca. 3 Sekunden \\
Geschätzte Gesamtzeit für den Abschluss der ersten Stufe: ca. 4 Minuten \\

\

\underline{Benötigte Zeit für Eskalationsstufe 2} \\
Zeit für die erste Stufe: ca. 4 Minuten \\
Hinlaufen zur Maschine: ca. 15 Sekunden \\
Entstörungsvorgang in der zweiten Stufe: ca. 40 Sekunden \\
Einschreiten der Zufallsvariable: ca. 3 Sekunden \\
Geschätzte Gesamtzeit für den Abschluss der ersten und der zweiten Stufe: ca. 5 Minuten \\

\

\underline{Benötigte Zeit für Eskalationsstufe 3 / Max. Zeit bei Erfolglosigkeit} \\
Zeit für die zweite Stufe: ca. 5 Minuten \\
Hinlaufen zur Maschine: ca. 15 Sekunden \\
Entstörungsvorgang in der dritten Stufe: ca. 40 Sekunden \\
Einschreiten der Zufallsvariable: ca. 3 Sekunden \\
Geschätzte Gesamtzeit für den Abschluss der ersten, zweiten und der dritten Stufe: ca. 6 Minuten \\























%\input{texes/eiderkl.tex}
\end{document}
